%% ModernCV Resume Template with Cover Letter

\documentclass[11pt,a4paper,roman]{moderncv}        % ModernCV class

% ModernCV theme
\moderncvstyle{classic}                            % Style: 'casual', 'classic', 'oldstyle', 'banking'
\moderncvcolor{green}                              % Color: 'blue', 'orange', 'green', 'red', 'purple', 'grey', 'black'

% Character encoding
\usepackage[utf8]{inputenc}                        % UTF-8 encoding
	\usepackage{ragged2e}
% Adjust page margins
\usepackage[scale=0.75]{geometry}

% Personal data
\name{D.}{Shankar}
\address{CSIR-National Institute of Oceanography}{Dona Paula, Goa}          % Optional, provide your address
\phone[mobile]{+91-832-2450-312}                    % Optional, mobile number
\email{shankar@csir.res.in, dshankar67@gmail.com}                        % Optional, email address

\begin{document}
	
	%-----       Letter       ---------------------------------------------------------
	% Recipient information
	\recipient{Editor}{Progress in Oceanography}
	\date{\today}
	\opening{Dear Editor}
	\closing{Yours sincerely,}


	% Letter body
	\makelettertitle
	\justifying
	We hope the letter finds you well. We are pleased to submit our manuscript titled "Intraseasonal to interannual variability of zooplankton biomass and standing stock inferred from ADCP backscatter in the eastern Arabian Sea" for your consideration in \textit{Progress in Oceanography}. The authors are Ranjan kumar Sahu, D.~Shankar, P.~Amol, S.~G.~Aparna, and D.~V.~Desai.
	
	Zooplankton, acts as critical intermediaries in marine food webs, influence both primary production and transfer energy to higher trophic levels. While conventional zooplankton sampling techniques have been used to study the seasonal variation, it has been largely limited by sampling strategy and less coverage, due to both natural and logistical limitations. The sparsity of data leads to a misrepresentation where a snapshot of an event is held as representative of the entire month or season, and further 
	an improper understanding in biomass variation. 
	
	Our study addresses these limitations using ADCP backscatter data from seven moorings along the slope off eastern Arabian Sea (EAS), Indian Ocean. We follow the methodology described in Aparna et al. 2022 (A22 from hereon), which was based on data from three moorings, to reliably convert the backscatter derived from ADCP to zooplankton biomass. In the present study, we extend their work by incorporating data from four additional moorings. The new data also allows us to validate the climatological biomass and standing stock for the three moorings that are shared between these two studies. It also enables us to analyse the transition occurring from a microbial-loop dominated northeastern Arabian Sea (NEAS) to classical food web dominated southeastern Arabian Sea (SEAS).
	
	While the seasonal cycle and climatology were discussed earlier in A22 at great detail, they hinted on the possibility of using ADCP backscatter data to study biomass variation on intraseasonal to interannual scale. The later was infeasible due to lack of sufficiently long record, but now made possible with the addition of three years of biomass data for the three overlapping moorings. 
	
	Our findings reveal several critical insights:
	
	\textbf{1. Intraseasonal Variability:} Intraseasonal component of zooplankton biomass variability is observed to be high, particularly in the SEAS. It is comparable to the variability in seasonal scale, indicating swift alteration in biomass lasting a few days to a few weeks. The range of variation in ZSS more than doubles from SEAS to NEAS, even if Chl-\textit{a} variation is reduced dramatically. This, along with observations on persistent high-frequency biomass variation helps us conclude the dominant food chain across EAS. 
	
	The spikes in biomass that last upto few days ($\sim$5~days) are present throughout the record, and the spike often exceed twice that of the error range of biomass estimation. The presence of spikes, irrespective of notable Chl-\textit{a} variation, indicates persistent presence of an alternative food chain that sustains the zooplankton biomass. 

	\textbf{2. Seasonal cycle:} The annual cycle is stronger in the NEAS, while intra-annual variability tends to be stronger at SEAS. Also, the later is much stronger than the former due to stronger asymmetry associated with seasonally reversing ocean currents. 

	\textbf{3 Climatology:} The comparison of climatology with A22 shows that basic features are in agreement, such as: 1) consistent pattern observed in the biomass contours that separate high biomass in the upper ocean from low biomass in the deeper layers, 2) the rise from the minimum during the summer monsoon occurs much faster than the gradual fall from the maximum during the winter monsoon, 3) lower zooplankton biomass during summer monsoon when the  phytoplankton biomass peaks in SEAS.
	
	\textbf{4. Interannual variability}: Observations shows presence of strong interannual component of variability at SEAS as seen from the decade long biomass time series data. Analysis reveals presence of Quasi-biennial oscillations, especially prominent off SEAS where the annual cycle is seemingly weak. Observing such type of variability is infeasible with traditional sampling methods. 
	
	The results suggest major implication on sampling methods used to study variation in zooplankton biomass and standing stock, at intraseasonal to interannual time scales. Conventionally, one or two snapshots from cruise based sampling are held as representative of the entire season, which is not the case. The short bursts in biomass that some time last a few days, can not be possibly captured in a cruise based sampling approach. Further, it is only with such a time series that it is possible to analyse the impact of climate modes such as ENSO or IOD. While the in-situ sampling has its own merits, the ADCP-backscatter derived biomass can provide much more information	if complimented appropriately by the former.
	
	We believe the manuscript is in line with the scope of the journal. We also confirm that this manuscript has not been published anywhere and is not under consideration by another journal. All authors listed on the manuscript are aware of the present submission.
	
	Thank you for your time and consideration.
	
	\makeletterclosing
	
\end{document}
