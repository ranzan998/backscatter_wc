%% ModernCV Resume Template with Cover Letter

\documentclass[11pt,a4paper,roman]{moderncv}        % ModernCV class

% ModernCV theme
\moderncvstyle{classic}                            % Style: 'casual', 'classic', 'oldstyle', 'banking'
\moderncvcolor{green}                              % Color: 'blue', 'orange', 'green', 'red', 'purple', 'grey', 'black'

% Character encoding
\usepackage[utf8]{inputenc}                        % UTF-8 encoding
	\usepackage{ragged2e}
% Adjust page margins
\usepackage[scale=0.75]{geometry}

% Personal data
\name{Ranjan Kumar}{Sahu}
\title{Curriculum Vitae}                          % Optional, remove if not wanted
\address{Physical Oceanography Division}{Dona Paula, Goa}          % Optional, provide your address
\phone[mobile]{+91-8322-450-400}                    % Optional, mobile number
\email{ranzan998@email.com}                        % Optional, email address

\begin{document}
	
	%-----       Letter       ---------------------------------------------------------
	% Recipient information
	\recipient{Editor}{Progress in Oceanography}
	\date{\today}
	\opening{Dear Editor}
	\closing{Yours sincerely,}


	% Letter body
	\makelettertitle
	\justifying
	\setlength{\parindent}{4em}We hope the letter finds you well. We are pleased to submit our manuscript titled "Intraseasonal to interannual variability of zooplankton biomass and standing stock inferred from ADCP backscatter in the eastern Arabian Sea" for your consideration in \textit{Progress in Oceanography}.
	
	\setlength{\parindent}{4em}Zooplankton, as critical intermediaries in marine food webs, influence both primary production and higher trophic levels. While conventional zooplankton sampling techniques have been used to study the seasonal variation, it has been largely limited by sporadic sampling and less coverage. The sparsity of data leads to lack of understanding in biomass variation. Our study addresses these limitations using ADCP backscatter data from seven moorings and we present the zooplankton biomass and standing stock estimation  along the slope of eastern Arabian sea (EAS), providing an unprecedented view of zooplankton dynamics across intraseasonal to interannual time scales. 
	
	Our findings reveal several critical insights:
	
	\textbf{1. Seasonal cycle:} The annual cycle is stronger in the north-EAS (NEAS), while Intra-annual variability tends to be dominant in south-EAS (SEAS).
	
	\textbf{2. Interannual variability}: Observations shows presence of strong interannual component of variability at SEAS as seen from the decade long biomass time series data. Observing such type of variability is infeasible with traditional sampling methods.
	
	\textbf{3. Intraseasonal Variability:} A strong intraseasonal component dominates zooplankton biomass, particularly in the SEAS. It is comparable to the variability in seasonal scale indicating swift alteration in biomass lasting a few days to a few weeks in response to the environmental drivers.
	
	\textbf{4. Implications of intraseasonal variability:} One of the major implication is on sampling methods used to study zooplankton seasonal cycle and variation. Conventionally, one or two snapshots from cruise based sampling are held as representative of the entire season, which is not the case. The second implication is presence or absence of coherence in biomass along the slope in EAS in intraseasonal scale, an indication of patchiness.
		
	The Arabian sea is much more productive than the Bay of Bengal even though both are near Indian subcontinent.  The ADCP backscatter derived zooplankton biomass and further standing stock is used to eliminate the shortcomings of the former method.
	
	We believe the manuscript is in line with the scope of the journal. We also confirm that this manuscript has not been published anywhere and is not under consideration by another journal.
	
	We look forward to your response and are happy to provide additional information if required. Thank you for your time and consideration.
	
	\makeletterclosing
	
\end{document}
