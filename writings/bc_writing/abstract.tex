\documentclass{article}

\linespread{2} 

{\Large 
	\title{Spatio-temporal variability of zooplankton standing stock in eastern Arabian Sea inferred from ADCP backscatter measurements }}
\author{Ranjan Kumar Sahu, P. Amol, D.V. Desai, S.G. Aparna, D. Shankar}


\makeatletter
\renewcommand{\maketitle}{
	\begin{titlepage}
		\begin{center}
			{\fontsize{14}{16}\selectfont\@title\par}
		\end{center}
		\null\vfill
		\noindent\justify
	\end{titlepage}
}
\makeatother

\begin{document}
	\maketitle
		The study focuses on the zooplankton standing stock in the eastern Arabian sea (EAS) and aims to understand its spatio-temporal variation using ADCP(acoustic Doppler current profiler) backscatter measurements. The ADCP moorings were deployed at multiple locations on the continental slope of the west coast of India; of which we have used data from October 2017 to January 2023. The ADCP (operating frequency 153.3 kHz) uses backscatter from or sediments or organisms such as copepods, ctenophores, salps and amphipods greater than 1 cm to calculate current profile. The conversion from backscatter to biomass is based on volumetric zooplankton sampling at the respective locations. Analysis of the data over 25-140 m shows that the backscatter and zooplankton biomass decrease from the upper ocean (215 mg m$^{-3}$ biomass contour) to the lower depths. Seasonal variation is noticed in the monthly climatology zooplankton standing stock (integral of the biomass over the 20-140m water column) along with change as we move to northward slope moorings in EAS. Complementary parameters (mixed layer depth, net primary production, Chl-a, sea surface temperature) is used to explain the processes leading to growth or decay in zooplankton biomass and on their migratory behaviour. Additionally, we have studied the effect of wind induced vertical mixing events. The findings of this research will contribute to a better understanding of the zooplankton dynamics in the EAS and provide valuable insights into the seasonal and annual cycles of zooplankton standing stock.
		
		
\end{document}