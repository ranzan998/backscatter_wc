\documentclass{article}
\usepackage{natbib}
\usepackage{multirow}
\usepackage{booktabs}
\usepackage{changepage}
\usepackage{caption} % For caption customization
\usepackage{lineno} % For line numbers
\usepackage{graphicx} % For including graphics
% Add line numbers to the document
\usepackage{geometry}

%\usepackage{hyperref}
\usepackage[colorlinks = true, linkcolor=blue, urlcolor=blue, citecolor=blue]{hyperref}
\usepackage[nameinlink]{cleveref}
\crefname{figure}{Fig.}{Figs.}
\linespread{2} 

{\Large 
	\title{Spatio-temporal variability of zooplankton standing stock in eastern Arabian Sea inferred from ADCP backscatter measurements }}
\author{Ranjan Kumar Sahu, P. Amol, D.V. Desai, S.G. Aparna,  D. Shankar}
\date{\today}
\begin{document}
	
	\maketitle
	\linenumbers
	\section*{Abstract}
	
We use acoustic Doppler current profiler (ADCP) backscatter measurements to map
the spatio-temporal variation of zooplankton standing stock in the eastern Arabian Sea (EAS). The ADCP moorings were deployed at seven locations on the continental slope off the west coast of India; we use data from October 2017 to December 2023. The 153.3 kHz ADCP uses backscatter from sediments or organisms such as copepods, ctenophores, salps and amphipods greater than 1 cm to calculate current profile. The backscatter is obtained from echo intensity using RSSI conversion factor after doing necessary calibrations. The conversion from backscatter to biomass is based on volumetric zooplankton sampling at the respective locations. Analysis of the data over 24 – 120 m shows that the backscatter and zooplankton biomass decrease from the upper ocean (215 $m\ g^{-3}$ biomass contour) to the lower depths. Changes are observed in the seasonal variation of the monthly climatology of zooplankton standing stock (integral of the biomass over 24 – 120 m water
column) as we move to poleward along the slope in EAS. The range of variation of standing stock is lowest at Kanyakumari, followed by Okha, which lie at the southern and northern boundary of the EAS, respectively. Complementary variables are used to explain the processes leading to growth or decay of zooplankton biomass.	
	\newpage
	\section{Introduction}
	\subsection{Background}
	Zooplankton plays a vital role in food web of pelagic ecosystem by enabling the hierarchical transport of organic matter from primary producers to higher trophic levels impacting the fish population and the carbon pump of the deep ocean \citep{ohman2001density,le2016global}. They are presumably the largest migrating organisms in terms of biomass \citep{hays2003review} which occurs in diel vertical migration (DVM). Zooplanktons depend not only on phytoplankton but other environmental parameters (e.g. Mixed layer depth, insolation, Oxygen, thermocline, nutrient availability, chlorophyll concentration and daily primary production). The biological productivity of the ocean is essentially connected with physics and chemistry \citep{subrahmanyan1959studiespart2, ryther1966primary, qasim1977biological, nair1970primary,banse1995zooplankton,mccreary2009biophysical, vijith2016consequences,amol2020modulation}. The dynamic ocean results in varying physico-chemical properties, leading to bloom and growth of planktons in favourable conditions. The changes are strongly influenced by the seasonal cycle in the North Indian Ocean (NIO; north of ~5 $^o$N of Indian Ocean). The eastern boundary of Arabian Sea contains the West India Coastal Current (WICC; \citep{patil1964hydrography,ramamirtham1965hydrography, banse1968hydrography,shetye1991coastal,mccreary1993numerical, shankar1997dynamics, shetye1998coastal, maheswaran2000upwelling, amol2014observed, chaudhuri2020observed,chaudhuri2021observed}) which reverses seasonally, flowing poleward (equatorward) during November to February (June to September). 
	
	The direct consequence of this reversal is the seasonal cycle of thermocline, oxycline and thickness of mixed Layer Depth (MLD) induced by upwelling favourable conditions in summer and downwelling favourable conditions in winter in eastern Arabian Sea (EAS). Further, the phytoplankton biomass and chlorophyll concentration changes with the season \citep{subrahmanyan1960studies, banse1968hydrography, levy2007basin, vijith2016consequences}. Upwelling in  summer monsoon leads to maximum chlorophyll growth in the entire EAS \citep{ banse1968hydrography, banse2000geographical, mccreary2009biophysical, hood2017biogeochemical}. During winter monsoon, the convective mixing induced winter mixed layer \citep{shetye1992does, madhupratap1996mechanism, levy2007basin, vijith2016consequences, shankar2016inhibition, keerthi2017physical} results in winter chlorophyll peak in northern EAS (NEAS) while the downwelling Rossby waves modulate chlorophyll along the southern EAS (SEAS) albeit limited to coast and islands \citep{amol2020modulation}. (For a detailed description on EAS division, please refer figure 1 of \citep{shankar2019role}.
	
	The zooplankton grazing peak is instantaneous with no time delay from peak phytoplankton production \citep{li2000determines}, but its population growth lags \citep{rehim2012dynamical, almen2020temperature} depending on its gestation period and other limiting aspects. While some studies suggest that the peak timing of zooplankton may not change in parallel with phytoplankton blooms \citep{winder2004climatic}, others indicate that lag exists between primary production and the transfer of energy to higher trophic levels \citep{brock1992interannual, brock1991phytoplankton}. A recent work \citep{aparna2022seasonal} had shown that peak zooplankton population may never occur even with a bloom in phytoplankton such as in SEAS, leading to the collapse of ecological models and succeeding food webs of higher trophic levels.  
	
	The conventional zooplankton measurements, where only few snapshot/s of the event is captured gives an incoherent or incomplete understanding in terms of spatio-temporal variation of zooplankton \citep{ramamurthy1965studies, piontkovski1995spatial, madhupratap1992zooplankton,madhupratap1996lack,wishner1998mesozooplankton} as much information is revealed by later studies \citep{jyothibabu2010re, vijith2016consequences, shankar2019role, aparna2022seasonal} using high resolution data. Calibrated acoustic instruments such as Acoustic Doppler Current Profiler (ADCP) along with relevant data can be utilised to understand small scale variability \citep{nair1999arabian, edvardsen2003assessing, smith2005mesozooplankton, smeti2015spatial, kang2024acoustic}, the complex interplay between the physico-chemical parameters and ecosystem \citep{jiang2007temporal, potiris2018acoustic, shankar2019role, aparna2022seasonal, nie2023influence}, the zooplankton migration \citep{inoue2016diel,ursella2018evidence, ursella2021diel} and their seasonal to annual variation \citep{jiang2007temporal, hobbs2021marine,liu2022seasonal, aparna2022seasonal}.
	
	\subsection{ADCP backscatter and zooplankton biomass}
	At present, there are two types of acoustic samplers: non-calibrated single frequency acoustic profiler such as ADCP or calibrated multi and mono frequency acoustic profilers such as zooplankton acoustic profiler (ZAP) and Tracor acoustic profiling system(TAPS). The use of acoustics as a proxy for zooplankton biomass estimation can be traced to \citep{pieper1971study, sameoto1977use} and earlier studies which used echograms to approximate the large-scale horizontal extents \citep{barraclough1969shallow}, and small scale vertical extent \citep{mcnaught1968acoustical}. The relationship between backscatter and the abundance and size of zooplankton was described by \citep{greenlaw1979acoustical} 
	wherein it was pointed out that single frequency backscatter can be used to estimate abundance if mean zooplankton size is known. This paved the way for use of single frequency acoustic profiler. A drastic increase in study temporal and spatial variation of zooplankton biomass using  backscatter-proxy came in 1990s by introduction of high frequency echo sounders, with studies \citep{flagg1989use, wiebe1990sound, batchelder00981, greene1998three, rippeth1998diur} methodically showing acoustic backscatter estimated zooplankton biomass in various shelf and slope locations around  North Atlantic, North pacific location. The foundation for further research that investigated the potential of acoustic backscatter from ADCPs and multi frequency echo sounders in assessing zooplankton biomass and comprehending zooplankton dynamics in diverse maritime habitats was established by these initial explorative experiments.
	
	 Acoustic backscatter and zooplankton biomass have been better understood as a result of technological and methodological developments over time. Net sampling augmented ADCP backscatter have been used to study DVM and the spatial and temporal variability of zooplankton biomass by \citep{cisewski2010seasonal,smeti2015spatial, guerra2019zooplankton} in different marine regions, such as the Southwestern Pacific, the Lazarev Sea in Antarctica and the Corsica Channel in the north-western Mediterranean Sea.	The zooplankton biomass variation in the Arabian sea has been studied during JGOFS programme in 1990s \citep{herring1998across, nair1999arabian, fielding2004biological, smith2005mesozooplankton}. However, their studies were limited to the cruise duration as vessel mounted ADCPs were predominantly used; hence long-term data was sparsely produced. The first such study to fully exploit the immense potential of ADCPs in EAS was carried out by \citep{aparna2022seasonal} using ADCP moorings deployed on continental slopes off the Indian west coasts \citep{amol2014observed, chaudhuri2020observed}.
	
	\subsection{Objective and scope of the manuscript}
	
	A network of ADCPs has been installed off the continental slope and shelf on the west coast of India. This ADCPs have enabled a rigorous view of intraseasonal to seasonal scale variability \citep{amol2014observed, chaudhuri2020observed}. Initially a network of 4 ADCPs (off Mumbai, Goa, Kollam and Kanyakumari) on continental slope, it has been extended to include 3 more moorings (off Okha from 2018, Jaigarh and Udupi from 2017). In the recent study \citep{aparna2022seasonal} have used ADCP moorings off  Mumbai, Goa and Kollam to explain the temporal variability of zooplankton biomass. The study showed that the zooplankton peaks (and troughs) is not only non-uniform in latitude but also heavily influenced by the oxygen minimum zone, MLD and the seasonal upwelling/downwelling conditions. Stark contrast in the phytoplankton bloom and subsequence  growth of zooplankton or the lack thereof was observed in the EAS regimes.


	We build upon the existing work by extending to include the newly incorporated ADCPs so as to have a better understanding in the latitudinal variation of zooplankton biomass in EAS. The paper is organized as follows; datasets and methods employed are described in detail in Section 2. Section 3 describes the observed climatology of zooplankton biomass and standing stock. A comparison is drawn to the results of previous studies at the overlapping mooring sites. Further, the seasonal cycle of zooplankton biomass and standing stock is discussed. The role of mixed layer depth, net primary production, sea surface temperature, wind forcing and circulation in determining the biomass is discussed in results section 4, with conclusion in section 5.
	
	\section{Data and methods}
	The  backscatter data from ADCP and the zooplankton samples collected from the periphery of mooring is described in this section. The methodology followed in processing ADCP data and estimation of backscatter and subsequently the zooplankton biomass is discussed. The backscatter derived from the echo intensity of the seven ADCP mooring deployed on the continental slope off the Indian west coast is the primary data we have use in this manuscript. The moorings details are summarised in \tablename{1}. In situ biomass data from volumetric zooplankton samples are used to validate and correlate with backscatter. The chlorophyll data is obtained from \href{https://data.marine.copernicus.eu/products}{marine.copernicus.eu}. In addition, we have used the monthly climatology of temperature and salinity \citep{chatterjee2012new} and the net primary productivity from MODIS (Moderate Resolution Imaging Spectroradiometer) and VIIRS (Visible Infrared Imaging Radiometer Suite) from global NPP estimates (\href{http://sites.science.oregonstate.edu/ocean.productivity}{http://sites.science.oregonstate.edu/ocean.productivity}). 
	
	\subsection{ADCP data and Backscatter estimation}
	The ADCPs were deployed on the continental slope off the Indian west coast (\cref{fig:fig1}). Initially a set of three ADCPs, it was gradually extended to four more sites to cover the entire EAS basin from Okha (22.26$^o$N) in north to Kanyakumari (6.96 $^o$N) in south. The other two ADCPs are  Jaigarh at central EAS (CEAS) and Udupi in the transition zone between CEAS \& SEAS. The extended moorings were deployed in October 2017, except for Kanyakumari which was deployed earlier as well but it wasn't included in earlier backscatter study. The moorings are serviced on yearly basis usually during October-November or in winter monsoon months. The ADCPs are of RD Instruments make, upward-looking and operate at 153.3 kHz. While utmost care is taken to position the instrument at  $\sim$ 200 m depth, yet for some deployments it's shallow or deeper owing to drift caused by floater buoyancy - anchor weight balance. Data was collected at hourly interval and the bin size was set to 4 m. The echoes at surface to 10 \% range (~20 m) means the data at these is rendered useless and is discarded from further use. 
	
	The procedure followed in processing of the ADCP data are described in \citep{amol2014observed} and \citep{mukherjee2014observed}. An addition to their methodology was to do depth correction to accommodate the vertical movement of ADCP buoys \citep{chaudhuri2020observed, mukhopadhyay2020observed}
    using data from pressure sensor mounted on the instrument. We have followed the methodology laid down in \citep{aparna2022seasonal} to derive the backscatter time series from ADCP echo intensity data which is discussed later paragraph. The gaps are filled using the grafting method of \citep{mukhopadhyay2020observed} once the zooplankton biomass time series is constructed.
    
	The primary objective of ADCP usage is to obtain vertical current profile at a point location. It is achieved by using the echo intensity received at the ADCP transducer. The instrument sensors doesn't directly give backscatter, as echo intensity is range independent. Range correction has to be performed before echo intensity (E) is converted to Backscatter (B). Received signal strength indicator (RSSI), also called the conversion factor (Kc) which is specific to a sensor is used along with the corresponding reference echo intensity (Er). It's important to state that for the same device Kc remains unchanged while Er varies over each subsequent deployment. The backscattering strength (in dB) is given by \citep{mullison2017backscatter}:
	
	$B = [C - L_{DBM}-P_{DBW}] + 2\alpha R + {10 log_{10}[(T_{TD}+273.16)R^2] } + {10log_{10} [10^{K_c(E-E_r)/10}-1]}$
	
	where $C$ is an empirical constant, $L_{DBM}$ is 10$log_{10}L$ where $L$ is the transmit pulse length in meters, $P_{DBW}$ is 10$log_{10}P$ ($P$ is  transmitted power in watts), $\alpha$ is the sound absorption coefficient of water (in $dB\ m^{-1}$),  $T_{TD}$ is the temperature (in $^o\ C$) at the depth of positioned instrument, $R$  is the slant range (in meters) from transducer to the scatterers and $E_r$ is	the reference level of $E$ taken in real-time (unit counts). $E_r$ in our case is taken from first (last) measured profile when the instrument is in air before (after) deployment (retrieval). The backscattering strength is referenced to ($4\pi m^{-1}$) \citep{deines1999backscatter, mullison2017backscatter}.  \citet{aparna2022seasonal} has discussed the relevance of each of the term to the total backscattering strength. Our analysis also suggests that the $\alpha$ does not affect the final results. 
	
	\subsection{Zooplankton data and estimation of biomass}
	The  zooplankton  samples were collected in the vicinity ($\sim$ 10 km) of ADCP mooring site twice; once prior retrieval and again post deployment of moorings so that there is overlap in the ADCP time instance and in situ zooplankton samples. The sampling is done at the mooring location during servicing cruises on board RV Sindhu Sankalp and RV Sindhu Sadhana (Table 2). Multi-plankton net (MPN) (100 $\mu m$ mesh size, 0.5 $m^2$ mouth area) was used to get samples in the pre-determined depth ranges; water volume filtered was calculated by the product of sampling depth range and the mouth area of net. The depth range and timing of sample collection was different throughout the MPN hauls. From 2020 onward, the depth-range was standardized to the bins of 0 - 25, 25 - 50, 50 - 75, 75 - 100, 100 - 150 (units are in meters). The collected zooplankton samples were then preserved in 5 \% formaldehyde solution until it's transferred to laboratory. To measure zooplankton wet weight accurately, the gelatinous forms/salps were separated. \citep{aparna2022seasonal} had reported the calanoid copepods, cyclopoid copepods, Poecilostomatoida, Harpacticoida, appendicularians, euphausids, ostracods, and
	chaetognaths as the major groups of zooplanktons contributing to the biomass of net samples from the mooring sites {\underline{this has to be updated to include later samples}}. 
	The backscatter obtained earlier is averaged in vertical corresponding to the specific MPN hauls for each site. The backscatter is linear regressed with respective biomass to establish their relationship, which has been demonstrated in numerous previous studies \citep{flagg1989use,heywood1991estimation,jiang2007temporal,aparna2022seasonal}. 
	
	We calculated the regression equation to be $y$ = 0.0203 $x$  + 4.01 and, which is well within the error range of the regression equation of \citep{aparna2022seasonal}, $y$ = (0.02±0.004) $x$ + (4.14±0.36) with a correlation of 0.53 (\cref{fig:fig2}). The correlation value in our case is 0.54; the minor difference is  due to higher number of data points (159) in the present study. 
	
	\subsection{Biomass time series and estimation of standing stock}
	
	The zooplankton biomass time series (\cref{fig:fig3}) is created from the above derived linear relationship: log$_{10}$(biomass) = $m$ * Backscatter + $k$, where $m$ is slope and $k$ is intercept. The time series shows the pattern of diel vertical migration (DVM) at all the mooring sites during dawn ($\sim$0600-0700 hours) and dusk ($\sim$1800-1900 hours). It is evident in earlier studies using backscatter \citep{ashjian2002distribution,smith2005mesozooplankton,inoue2016diel,ursella2018evidence} and in situ zooplankton data \citep{padmavati1998vertical}. The implication of DVM is a higher biomass at surface during the night as zooplankton feeds and a lower biomass at daytime as they descend to subsurface depths. The overall biomass over the time period of a day may vary but the DVM doesn't affect the seasonal variation as shown by (\citep{jiang2007temporal,aparna2022seasonal}). Since our goal is to study the seasonal variation, delineating the daily biomass is sufficient. The biomass time series is discussed in section \textbf{name the section 3.1?}.
	
	The standing stock is determined by taking the depth integral of biomass over the water column. To maintain the consistency of standing stock estimation, only those deployments that doesn't lack data at any depth in the entire range of 24 - 120 m are considered for analysis. The lack of data in the above mentioned depth range is due to deviation in positioning of ADCP sensor in the water column. A swift alteration in bathymetry along the continental slope implies that the mooring might anchor at a different depth than planned, hence a change in the predicted position of ADCP. This leads to gap in data at few mooring sites for some year. For example, for the northern-most mooring at Okha, data is not available for the entire upper 120 m depth for the second deployment. Also at Jaigarh, where the surface to $\sim$60m data (in 3rd deployment) and Kollam, where 80 m and below (in 4th deployment) is unavailable and hence discarded from standing stock estimation. There are few deployments where no data or bad data was recorded e.g, at Udupi (4th deployment) and Kanyakumari (6th deployment). The seasonal cycle of standing stock for 24 - 120 m available data is explored in section \textbf{name of the section, 3.2?}.
	
	\subsection{Chlorophyll and net primary productivity data}
	Previous study based on ADCP data of EAS \citep{aparna2022seasonal} have used SeaWIFS based chlorophyll data for comparison with climatology of zooplankton standing stock (ZSS). The SeaWIFS was at its end of service in 2010, hence we use new chlorophyll product. The present study has been conducted using Global Ocean Colour, biogeochemical, L3 data obtained from the  \href{https://doi.org/10.48670/moi-00280}{E.U. Copernicus Marine Service Information}. The daily data is available at a spatial resolution of 4 km. 

	While chlorophyll is used to compare with the variation in climatology of zooplankton standing; the growth efficiencies of zooplankton are directly linked to primary production levels, emphasizing the interconnectedness between primary producers and consumers in marine food webs \citep{Friedland.2012}. In their study, \citep{aparna2022seasonal} has emphasized on the collapse of the predator-prey relationship between zooplankton-phytoplankton using climatological data. We showcase their interdependency or the lack thereof using net primary productivity models.
	Moderate Resolution Imaging Spectroradiometer (MODIS) based net primary productivity (NPP) data at a resolution of 0.16$^o$ x 0.16$^o$ was obtained from Oregon State University. They have employed three different schemes to obtain NPP from Chlorophyll concentration. Those are discussed below in brief. The first is Vertically Generalized Production Model (VGPM). The NPP (a rate term) is to be derived from chlorophyll (a standing stock) using chlorophyll-specific assimilation efficiency for carbon fixation. The single biggest unknown in all models based on chlorophyll is how this rate term is described. VGPM considers the primary productivity to be dependent on day length and maximum daily NPP within a water column. The second is Carbon-based Productivity Model (CbPM) which NPP to phytoplankton carbon biomass and growth rate. The third is Carbon, Absorption, and Fluorescence Euphotic-resolving (CAFE) mode; first described by \citep{silsbe2016cafe} takes various other factors into NPP calculations. We explore these NPP models and try to explain the variation in ZSS.
	 
	\section{Climatology of zooplankton biomass and standing stock}
	The previous study of zooplankton in EAS based on ADCP backscatter was consisting of three sites: Mumbai in NEAS, Goa in CEAS and Kollam in SEAS. The extended mooring sites are at Okha at NEAS, Jaigarh at CEAS, Udupi in the transition zone of CEAS \& SEAS and Kanyakumari at SEAS is the southern most location in our study area. 

	ADCP data from three mooring sites were analysed from 2012 to 2020 in \citep{aparna2022seasonal}. They have fine-tuned the methodology to obtain backscatter and estimated zooplankton biomass from it using in situ volumetric zooplankton biomass data. A comparison is made in later paragraphs, since the methodology remains same in the current study and new time series data is available. The monthly climatology of biomass and ZSS is computed for all locations having valid data in 24 - 120 m depth range (\cref{fig:fig4}).
	 
% ommited out
%	The high biomass regime in the upper ocean and low biomass regime in deeper depths is differentiated using the 215 $mg\ m^{-3}$ biomass contour. For simplicity, this biomass contour is abbreviated to be z215 and its depth is denoted as D215 henceforth; region lying above this contour is the upper ocean biomass. The choice of 215 $mg\ m^{-3}$ isn't abrupt; it is carefully chosen to accommodate the seasonal variation, as a shift to biomass contour lower than the z215 would be unviable as our data is only till 140 m depth. A higher biomass contour would lead to inferior view of the seasonal cycle such as in the case of Kanyakumari and Okha where D215 is often low enough to reach $\sim$20 - 30 m depths. The climatology of zooplankton biomass at different mooring location is discussed at locations northward starting from southern mooring location (Fig. 4).
	

	The high biomass regime in the upper ocean and low biomass regime in deeper depths is differentiated using a biomass contour: 215 $mg \ m^{-3}$ off Mumbai, Goa and Kollam; 200 $mg \ m^{-3}$ off Jaigarh and Udupi; 175 $mg \ m^{-3}$ off Okha and Kanyakumari. For simplicity, this biomass contour is abbreviated to be z215, z200 \& z175 and its depth is denoted as D215, D200 \& D175, respectively. The choice of biomass contour isn't abrupt; firstly, it is carefully chosen to accommodate the seasonal variation, as a shift to biomass contour lower than the z215 would be unviable as our data is only till 140 m depth as in the case of Kollam. A higher biomass contour would lead to inferior view of the seasonal cycle such as in the case of Kanyakumari and Okha where 215 $mg \ m^{-3}$ biomass contour is often low enough to reach $\sim$20 - 30 m depths, hence z175 is chosen here. Secondly, it allows us to link the seasonal variation of biomass to the physico-chemical properties.
	
	The climatology of zooplankton biomass (\cref{fig:fig4}) is discussed at locations northward starting from southernmost mooring site off Kanyakumari. 
	
%	\subsection{Southern EAS}
%	At Kanyakumari, during March to May, the zooplankton biomass is restricted to top layer ($\sim$ 30 m) as inferred from D215 (Fig. 4 g1). A thin layer of low biomass is seen from mid-May to late-June which divides the upper layer biomass to three distinct regions; a low biomass layer sandwiched between layers of higher biomass. With the advent of summer monsoon, the depth of 23 $^o$ C isotherm (henceforth D23) shallows along-with oxycline and a rise in biomass is observed. The D215 deepens reaching as deep as 60 m during June to September. It is sustained throughout the summer monsoon and begins to decline in October-November. Contrary to May-June biomass, we see a thin layer of high biomass ($\sim$ 35-50 m) in this period. This could be due to advection of biomass carried by the monsoon currents. During winter monsoon; the D215 shifts to shallower depths. A gradual increase is seen in the chlorophyll biomass starting from April and the peak is attained in June (Fig. 4 g2). The ZSS is increased in June, however the growth is minimal. There is almost no seasonal variation in ZSS off Kanyakumari (seasonal ZSS range, 1.56 $gm\ m^{-2}$) as compared to the ZSS variation at the nearest northern mooring site off Kollam (seasonal ZSS range, 4.09 $gm\ m^{-2}$), where a strong seasonal cycle is observed and the D215 is deeper for any given month. Off Kollam, higher biomass is present in the larger portion of water column and the D215 is at $\sim$ 110 m during Mar-May (Fig. 4. f1). Similar to Kanyakumari, the decrease in biomass with depth is gradual and there is presence of biomass below z215. The D215 begins to shallow with progressing summer monsoon. During this period, a sharp decrease is seen in the D23 ($\sim$ 60 m in June to September) while the oxycline (2.1 $ml \ L^{-1}$) overshoots the thermocline (Fig. 4. f1). A steep rise in chlorophyll biomass is seen off Kollam and its peak is attained in August. The ZSS declines in the same period and reaches a minimum when the chlorophyll biomass is at its peak. The chlorophyll biomass decreases rapidly in the following months, while the ZSS increases and a maximum is seen during October. This feature was earlier reported by \citep{aparna2022seasonal} showing disproportionate interaction between zooplankton and phytoplankton. This begs the question of existing understanding of predator - prey relationship in a local-scale ecological system. %$\underline{add \ explanation \ in \ later para, put reference here.}$% 	A similar feature is seen further north mooring site albeit with a relatively weaker zooplankton biomass, off Udupi which sits at the transition zone of SEAS \& CEAS. The peak of chlorophyll and minimum of ZSS occurs in September (Fig. 4 e2) which one month later than off Kollam. The 2.1 $ml \ L^{-1}$  Oxygen contour overshoots thermocline, however it reaches to a much shallow depth of $\sim$ 20 m. The D215 closely follows D23 from late June to November, and it is shallower for rest of the year. 
	 
	\subsection{Southern EAS}
	At Kanyakumari, with the advent of summer monsoon, the depth of 23 $^o$ C isotherm (henceforth D23) shallows along-with oxycline (marked by 2.1 $ml \ L^{-1}$) and a rise in biomass is observed (\cref{fig:fig4} g1). The z175 is shallower during June to September and the zooplankton biomass is comparatively higher than rest of the year. The D175 deepens starting from October and the relatively high biomass in water column is maintained till late December. However, this increase in D175 isn't reflected as an increase in ZSS because of low biomass in the entire water column. A gradual increase is seen in the chlorophyll biomass starting from April and the peak is attained in June (\cref{fig:fig4} g2). The ZSS is increased in June, however the growth is minimal. There is almost no seasonal variation in ZSS off Kanyakumari (seasonal ZSS range, 1.56 $gm\ m^{-2}$) as compared to the ZSS variation at the nearest northern mooring site off Kollam (seasonal ZSS range, 4.09 $gm\ m^{-2}$), where a strong seasonal cycle is observed and the D215 is deeper for any given month.
	
	Off Kollam, a higher biomass is present in the larger portion of water column and the D215 is at $\sim$ 110 m during Mar-May (\cref{fig:fig4} f1). Similar to z175 off Kanyakumari, the decrease in biomass with depth is subtle below z215. The D215 begins to shallow with progressing summer monsoon. During this period, a sharp decrease is seen in the D23 ($\sim$ 60 m in June to September) while the oxycline (1.7 $ml \ L^{-1}$) overshoots the thermocline (Fig. 4. f1). A steep rise in chlorophyll biomass is seen off Kollam and its peak is attained in August (\cref{fig:fig4} f2). The ZSS declines in the same period and reaches a minimum when the chlorophyll biomass is at its peak. The chlorophyll biomass decreases rapidly in the following months, while the ZSS increases and a maximum is seen during October. This feature was earlier reported by \citep{aparna2022seasonal} showing disproportionate interaction between zooplankton and phytoplankton. It begs the question of existing understanding of predator - prey relationship in a local-scale ecological system. A similar feature is seen further north, off Udupi which sits at the transition zone of SEAS \& CEAS, albeit with a relatively weaker zooplankton biomass. The peak of chlorophyll and minimum of ZSS occurs in September (\cref{fig:fig4} e2) which is one month later than off Kollam. The 2.1 $ml \ L^{-1}$  oxygen contour overshoots thermocline, however it reaches to a much shallow depth of $\sim$ 20 m during July to October. The D200 closely follows D23; with the gradual shallowing from March onward reaching $\sim$ 60 m in September and a steep decline afterwards till November (\cref{fig:fig4} e1). Decrease in biomass with depth is moderate in comparison to Kollam.
	

	\subsection{Central EAS}
	Off Goa, the D215 seasonal trend is similar to Udupi and is entirely restricted by D23 \& oxycline that closely follows it. During March-May, the D215 is at $\sim$ 100 m which shallows with onset of summer monsoon; the chlorophyll biomass increases during this period and the maximum occurs in August after which the chlorophyll biomass and ZSS both decrease in September. Although we witness an increase in chlorophyll biomass in October, the D215 is restricted to the $\sim$ 50 m in this period (fig. 4d1) and the ZSS is at it minimum (\cref{fig:fig4} d2) similar to what is observed off Udupi and Kollam. The ZSS rapidly increases and reaches its maximum in January, sustained till March and then gradually declines. Unlike the previous locations, the biomass off Goa decreases rapidly below the z215 as reported earlier \citep{aparna2022seasonal}, reaching as low as 60 $mg \ m^{-3}$ during June to September at 130 m (\cref{fig:fig4} d1).
	 
%The ZSS off Jaigarh is identical but stronger to that of off Goa, owing to an higher biomass above z200 and the comparatively deeper D200. The D200 is restricted by D23 \& oxycline for most of the year and it only exceeds during October-December(Fig. 4c1). From the ZSS maximum in February, it steadily decreases and attains a minimum in September (coincides with lower D215), a rapid rise is seen in the following months. What's intriguing is a presence of strong seasonal cycle in ZSS off Jaigarh (7.52 $gm\ m^{-2}$, highest among all locations) although the seasonal variation in Chlorophyll biomass (Fig. 4c2) is visibly non-existent (0.55 $mg\ m^{-3}$, lowest among all locations). This is an exact opposite scenario of Kanyakumari mooring site, where an insignificant seasonal variation in ZSS (1.56 $gm\ m^{-2}$) is seen even though the chlorophyll biomass varies strongly (1.61 $mg\ m^{-3}$). 

	
	The ZSS off Jaigarh is identical but stronger to that of off Goa, owing to an higher biomass above z200 and the comparatively deeper D200 (\cref{fig:fig4} c2). The D200 follows D23 \& oxycline for most of the year and it only exceeds during October-December (\cref{fig:fig4} c1).  From the ZSS maximum in February, it steadily decreases and attains a minimum in September (coincides with lower D200), a rapid rise is seen in the following months. What's intriguing is a presence of strong seasonal cycle in ZSS off Jaigarh (7.52 $gm\ m^{-2}$, highest among all locations) although the seasonal variation in chlorophyll biomass (\cref{fig:fig4} c2) is visibly non-existent (0.55 $mg\ m^{-3}$, lowest among all locations). This is an exact opposite scenario of Kanyakumari site, where an insignificant seasonal variation in ZSS (1.56 $gm\ m^{-2}$) is seen even though the chlorophyll biomass varies strongly (1.61 $mg\ m^{-3}$). 
	%The reason for the higher ZSS is looked upon in later section.
		
	Starting from Kollam (\cref{fig:fig4} f1) and moving northward to Jaigarh (\cref{fig:fig4} c1), we see that the core of high zooplankton biomass gradually shifts from summer monsoon (off Kollam) to winter monsoon (off Jaigarh), with the transition of upper ocean zooplankton biomass happening along Udupi and Goa. 
	 
	\subsection{Northern EAS}
	Further north of Jaigarh, off Mumbai the D215 follows a similar pattern i.e, a deeper D215 in December to early April, resulting in a higher ZSS in the same period (Fig. 4b2). 	 The D23 off Mumbai follows D215 and the oxycline follows an erratic pattern, reaching depths $>$ 140 during January to March (Fig. 4b1); when a higher biomass is observed above z215. The chlorophyll biomass shows seasonal variation albeit lower than the SEAS counterpart. Its peak occurs in August, then decreases rapidly and increases from October onward maintaining the biomass at 0.5 $mg\ m^{-3}$ till March. In zooplankton biomass climatology, during September-October a thin layer of low biomass regime is see at depths $\sim$30 - 40 m, combined with shallow D215 leading to the ZSS minimum. The ZSS increases rapidly from its minima in October in the following month as the D215 deepens and the maximum occurs in February. The chlorophyll biomass decreases from March and a gradual decrease in ZSS is seen till July, after which the ZSS basically flattens even though the chlorophyll increases. 
	
%	At the northernmost site of EAS i.e, off Okha, a noticeable feature is deeper oxycline, exceeding $>$ 140m from February to late May while the D215 is at a shallow depth of 80m for the same period, while D23 lies in between at $\sim$ 100m. The biomass above z215 is much weaker compared to Mumbai as seen in the zooplankton biomass climatology owing to a shallower D215 which leads to a lower ZSS. From April onward a low biomass layer is seen at the upper ocean $\sim$20-40 m which extends till Late November. For this period, a high zooplankton biomass layer is sandwiched between two low biomass layer, similar to Mumbai zooplankton biomass in September-October.  There's two chlorophyll biomass maxima off Okha; one in February \citep{keerthi2017physical} and the other during August in summer monsoon \citep{levy2007basin}. The ZSS remains flat in June to September although the chlorophyll biomass increases period. Afterwards, ZSS gradually increases and attains its maximum in February same as the chlorophyll biomass. The ZSS sustains this maximum till March, declines rapidly in April and then gradually till July.  
	 
	 At the northernmost site of EAS i.e, off Okha, a noticeable feature is a much higher oxygen in upper ocean. The biomass above z200 is much weaker (\cref{fig:fig4} a1) compared to Mumbai as seen in the zooplankton biomass climatology which leads to a relatively lower ZSS (\cref{fig:fig4} a2). The D200 shallows from February (coinciding with ZSS maximum) to it's minimum in August,  remains visibly flat till September and then increases steadily till December and rapidly afterwards. There's two chlorophyll peak off Okha; one in February \citep{keerthi2017physical} and the other during August in summer monsoon \citep{levy2007basin}. The ZSS remains flat in summer monsoon period i.e, June to September, although the chlorophyll biomass increases in this time. Afterwards, ZSS gradually increases and attains its maximum in February same as the chlorophyll biomass. The ZSS sustains this maximum till March, declines rapidly in April and then gradually till July.
	 
	\subsection{Comparison to previous result}	 
	A comparison with the zooplankton biomass and standing stock climatology of previous work \citep{aparna2022seasonal} is made in this section for the locations of Mumbai, Goa and Kollam. In the previous study data from 2012 to 2020 is used, while the present study includes data 2017 to 2023.
	 
	The D215 is shallower at all locations and as a result a lower ZSS is seen in the climatology of the present study (Fig. 5). The difference in D215 is prominent off Goa; while in the previous climatology (Fig. 5b1), the D215 is deeper and lies along D23, in the present climatological data (Fig. 5b2), The D215 is shallower and lies $\sim$ 20 - 40 m above the D23 during January to April; and a relatively lower biomass is present above z215. This goes same for the biomass off Mumbai (Fig. 5a1 \& 5a2) i.e, a comparatively shallow D215 and lower ZSS in comparison with \citep{aparna2022seasonal}. Off Kollam, a higher biomass is observed from May to June in previous study, while in the present study, along with May to June a higher biomass is seen from September to November(Fig 5c2) which is reflected as a minima of ZSS occurring in August (Fig. 5d2). The higher ZSS on either side to this minima is less pronounced in previous data. Chlorophyll biomass shows stronger peak for all locations in August in present study, when the zooplankton-phytoplankton relationship discrepancy is observed off Kollam similar to results reported in previous climatology.
	 	
	 	
	 
	 
	\subsection{The seasonal cycle of biomass}
	The seasonal cycle of zooplankton of these seven location is described in this section.
	   
	A preliminary analysis of the biomass time series in daily and monthly averaged scale shows that the biomass decreases with increasing depth (Fig. 3) at all the seven locations. 
	  
	The rate of biomass decrease with depth, roughly defined as the difference between the mean biomass at 40 m  and 105 m depth, is highest off Jaigarh and Mumbai as it has higher biomass in upper ocean (Fig. 3,c2,b2). This is followed by CEAS locations Goa and Udupi. While the rate of biomass decrease is lower off Kollam for 2017 to 2020. The rate of decrease is lowest off Okha and Kanyakumari. Written in the order of their rate of decrease from the difference of biomass: Jaigarh (96 $mg\ m^{-3}$), Mumbai (91 $mg\ m^{-3}$), Okha (79 $mg\ m^{-3}$), Udupi (78 $mg\ m^{-3}$), Kollam(73 $mg\ m^{-3}$), Goa (72 $mg\ m^{-3}$) and Kanyakumari (39 $mg\ m^{-3}$). The weaker decline in zooplankton biomass with respect to the depth at Okha (Fig. 3.a1,a2) at NEAS is agreeing with earlier reported data \cite{madhupratap2001mesozooplankton,smith2005mesozooplankton,wishner1998mesozooplankton} where oxygen deficit at is thought to be the cause. The sites at SEAS, especially off Kanyakumari and 2017 to 2020 off Kollam also have weaker decline \citep{madhupratap2001mesozooplankton, aparna2022seasonal}. However, after 2020 the rate of decline in biomass with depth off Kollam is similar to that off Mumbai in stark contrast to its previous years. This is due to a strong bloom in these years also seen at other locations. This high growth led to increase in biomass in the entire water column(Fig. 3.f1,f2).
	 
	Analysis of the D215 shows strong seasonality at the CEAS moorings off Goa and Jaigarh, with low variation in biomass in its upper ocean, with shallow (deeper) D215 in summer (winter) monsoon. This is followed by the moorings at transition zone off Mumbai (NEAS) and Udupi (SEAS). The 
	 
	{\underline{include wavelet analysis results}}

	\newpage	 

	%{\underline {make table for the, 40 and 104 mean biomass, the difference, their std}}
	%\textbf{note to self} edit table 1; Jaigarh last mooring.
	\newpage
	
	
	
	
	
	
	
	
	
	
\linespread{1.5}	
{\footnotesize 	\bibliographystyle{plainnat} % Choose a bibliography style
	\bibliography{bs_citations} % Specify your .bib file
}	
\newpage
\newgeometry{top=1in, bottom=1in} 

\linespread{1} 	
\begin{table}[htbp]

	{\footnotesize

		\captionsetup{justification=justified,font=footnotesize,skip=0.05\baselineskip} % Adjust the spacing above and below the caption
		\caption{ADCP deployment details at the locations. The temporal resolution is 1 hour, bin size(vertical resolution) 4 m. All ADCPs are operated at 153.3 kHz. The moorings are at a water column depth of ~950 - 1200 m on the continental slope and are serviced on yearly basis according to ship availability. The 6th column consists of Reference echo intensity (Er) for each beam, while the 7th column contains the corresponding RSSI conversion factor \citep{deines1999backscatter}.}
		\begin{adjustwidth}{0in}{0in} 
			\begin{tabular}{ccccccc}
				
				\toprule
				\multicolumn{1}{c}{}        & \multicolumn{2}{c}{Date}                                       & \multicolumn{2}{c}{Depth}                                                              & &          \\ 
				\midrule
				\multicolumn{1}{c}{\begin{tabular}[c]{@{}c@{}} Station \\ (Position; $^o$E,$^o$N) \end{tabular}} & \multicolumn{1}{c}{Deployment} & \multicolumn{1}{c}{Recovery} & \multicolumn{1}{c}{Ocean} & \multicolumn{1}{c}{ADCP} & \multicolumn{1}{c}{Er} & \multicolumn{1}{c}{Kc} \\
				\midrule
				\multirow{4}{*}{\begin{tabular}[c]{@{}c@{}} Okha\\ (67.47, 22.26)\end{tabular}}         & 01/10/2018                      & 01/12/2019                    & 996                        & 118                       & 37                          , 37                          , 37                          , 36 &                           0.42                        , 0.44                        , 0.42                        , 0.43                        \\
				& 01/12/2019                      & 04/12/2020                    & 1166                       & 312                       & 39                          , 36                          , 38                          , 36                          & 0.42                        , 0.44                        , 0.42                        , 0.43                        \\
				& 04/12/2020                      & 08/03/2022                    & 1021                       & 144                       & 41                          , 37                          , 38                          , 37                          & 0.42                        , 0.44                        , 0.42                        , 0.43                        \\
				& 08/03/2022                      & 01/01/2023                    & 1019                       & 142                       & 37                          , 38                          , 39                          , 36                          & 0.42                        , 0.44                        , 0.42                        , 0.43                        \\
				\midrule
				\multirow{5}{*}{\begin{tabular}[c]{@{}c@{}} Mumbai \\ (69.24, 20.01)\end{tabular}}        & 09/11/2017                      & 29/09/2018                    & 1025                       & 150                       & 36                          , 34                          , 39                          , 42                          & 0.40                        , 0.40                        , 0.40                        , 0.40                        \\
				& 29/09/2018                      & 29/11/2019                    & 1122                       & 125                       & 35                          , 36                          , 39                          , 42                          & 0.40                        , 0.40                        , 0.40                        , 0.40                        \\
				& 29/11/2019                      & 02/12/2020                    & 1143                       & 164                       & 37                          , 34                          , 39                          , 43                          & 0.40                        , 0.40                        , 0.40                        , 0.40                        \\
				& 02/12/2020                      & 06/03/2022                    & 1125                       & 142                       & 36                          , 34                          , 39                          , 42                          & 0.40                        , 0.40                        , 0.40                        , 0.40                        \\
				& 07/03/2022                      & 02/01/2023                    & 1103                       & 158                       & 37                          , 34                          , 40                          , 43                          & 0.40                        , 0.40                        , 0.40                        , 0.40                        \\
				\midrule
				\multirow{5}{*}{\begin{tabular}[c]{@{}c@{}} Jaigarh \\ (71.12, 17.53)\end{tabular}}       & 27/10/2017                      & 27/09/2018                    & 1039                       & 198                       & 32                          , 35                          , 33                          , 32                          & 0.45                        , 0.45                        , 0.45                        , 0.45                        \\
				& 27/09/2018                      & 30/10/2019                    & 1032                       & 164                       & 32                          , 35                          , 33                          , 31                          & 0.45                        , 0.45                        , 0.45                        , 0.45                        \\
				& 03/11/2019                      & 30/11/2020                    & 1142                       & 264                       & 32                          , 36                          , 33                          , 32                          & 0.45                        , 0.45                        , 0.45                        , 0.45                        \\
				& 30/11/2020                      & 05/03/2022                    & 1099                       & 119                       & 33                          , 36                          , 34                          , 32                          & 0.45                        , 0.45                        , 0.45                        , 0.45                        \\
				& 03/04/2022                      & 26/06/2022                    & 1120                       & 136                       & 68                          , 71                          , 69                          , 66                          & 0.45                        , 0.45                        , 0.45                        , 0.45                        \\
				\midrule
				\multirow{5}{*}{\begin{tabular}[c]{@{}c@{}} Goa\\ (72.74, 15.17)\end{tabular}}          & 03/10/2017                      & 25/09/2018                    & 1000                       & 174                       & 35                          , 37                          , 34                          , 35                          & 0.44                        , 0.44                        , 0.40                        , 0.41                        \\
				& 25/09/2018                      & 16/10/2019                    & 969                        & 145                       & 38                          , 36                          , 36                          , 34                          & 0.44                        , 0.44                        , 0.40                        , 0.41                        \\
				& 16/10/2019                      & 29/11/2020                    & 966                        & 143                       & 44                          , 38                          , 36                          , 43                          & 0.44                        , 0.44                        , 0.40                        , 0.41                        \\
				& 29/11/2020                      & 03/03/2022                    & 985                        & 157                       & 35                          , 40                          , 35                          , 38                          & 0.44                        , 0.44                        , 0.40                        , 0.41                        \\
				& 03/03/2022                      & 05/01/2023                    & 984                        & 159                       & 35                          , 38                          , 35                          , 34                          & 0.44                        , 0.44                        , 0.40                        , 0.41                        \\
				\midrule
				\multirow{4}{*}{\begin{tabular}[c]{@{}c@{}} Udupi \\ (74.04, 12.5)\end{tabular}}         & 05/10/2017                      & 06/10/2018                    & 1028                       & 176                       & 44                          , 46                          , 29                          , 35                          & 0.45                        , 0.45                        , 0.45                        , 0.45                        \\
				& 06/10/2018                      & 18/10/2019                    & 1027                       & 179                       & 32                          , 38                          , 30                          , 36                          & 0.45                        , 0.45                        , 0.45                        , 0.45                        \\
				& 18/10/2019                      & 11/12/2020                    & 1018                       & 168                       & 33                          , 37                          , 31                          , 38                          & 0.45                        , 0.45                        , 0.45                        , 0.45                        \\
				& 11/03/2022                      & 06/01/2023                    & 1036                       & 155                       & 31                          , 32                          , 32                          , 33                          & 0.45                        , 0.45                        , 0.45                        , 0.45                        \\
				\midrule
				\multirow{5}{*}{\begin{tabular}[c]{@{}c@{}} Kollam \\ (75.44, 9.05)\end{tabular}}        & 07/10/2017                      & 08/10/2018                    & 1174                       & 200                       & 43                          , 55                          , 45                          , 43                          & 0.49                        , 0.50                        , 0.49                        , 0.50                        \\
				& 08/10/2018                      & 20/10/2019                    & 1160                       & 123                       & 49                          , 62                          , 46                          , 46                          & 0.49                        , 0.50                        , 0.49                        , 0.50                        \\
				& 20/10/2019                      & 13/12/2020                    & 1209                       & 176                       & 52                          , 61                          , 54                          , 55                          & 0.49                        , 0.50                        , 0.49                        , 0.50                        \\
				& 13/12/2020                      & 13/03/2022                    & 1129                       & 91                        & 49                          , 51                          , 46                          , 47                          & 0.49                        , 0.50                        , 0.49                        , 0.50                        \\
				& 13/03/2022                      & 08/01/2023                    & 1149                       & 164                       & 41                          , 48                          , 43                          , 41                          & 0.49                        , 0.50                        , 0.49                        , 0.50                        \\
				\midrule
				\multirow{6}{*}{\begin{tabular}[c]{@{}c@{}} Kanyakumari \\ (77.39,6.96)\end{tabular}}   & 16/11/2016                      & 08/10/2017                    & 1096                       & 252                       & 37                          , 36                          , 37                          , 37                          & 0.42                        , 0.44                        , 0.42                        , 0.43                        \\
				& 08/10/2017                      & 10/10/2018                    & 1055                       & 181                       & 32                          , 34                          , 38                          , 35                          & 0.45                        , 0.45                        , 0.45                        , 0.45                        \\
				& 10/10/2018                      & 22/10/2019                    & 1075                       & 180                       & 36                          , 34                          , 39                          , 36                          & 0.45                        , 0.45                        , 0.45                        , 0.45                        \\
				& 22/10/2019                      & 14/12/2020                    & 1060                       & 167                       & 33                          , 35                          , 36                          , 35                          & 0.45                        , 0.45                        , 0.45                        , 0.45                        \\
				& 14/12/2020                      & 14/03/2022                    & 1184                       & 287                       & 34                          , 36                          , 36                          , 35                          & 0.45                        , 0.45                        , 0.45                        , 0.45                        \\
				& 14/03/2022                      & 10/01/2023                    & 1069                       & 172                       & 33                          , 36                          , 42                          , 36                          & 0.45                        , 0.45                        , 0.45                        , 0.45                       
				\\ 
				\bottomrule
			\end{tabular}
		\end{adjustwidth}
	}
	
	
\end{table}
\restoregeometry

\newpage

\begin{table}[htbp]
	
	{\footnotesize
		\captionsetup{justification=justified,font=footnotesize,skip=0.05\baselineskip,width=\textwidth} % Adjust the spacing above and below the caption
		\caption{\newline Volumetric samples of zooplankton of various stations. The tags corresponds to cruise and particular station. The sampling depth range is standardised for later years for bin range of 0-25m, 25-50m, 50-75m, 75-100m, 100-150m}
		\begin{adjustwidth}{0in}{0in} 
			\begin{tabular}{ccccccc}
				\toprule
				Sample number & Tag & Lat($^o$N)    & Lon($^o$E)   & Date & Time (IST) & Sampling depth range (m)      \\
				\midrule
				1-3         & G1  & 15.18      & 72.79      & 25 Sep 18                 & 452        & 50–25, 100–50, 150–100        \\
				4-6         & G2  & 15.16      & 72.71      & 25 Sep 18                 & 2108       & 50–25, 100–50, 150–100        \\
				7-10        & G2  & 15.16      & 72.71      & 25 Sep 18                 & 2137       & 40–20, 60–40, 80–60, 100–80   \\
				11-14       & J1  &            &            & 26 Sep 18                 & 2000       & 40–20, 60–40, 80–60, 100–80   \\
				15-17       & J2  &            &            & 27 Sep 18                 & 2000       & 50–25, 100–50, 150–100        \\
				18-21       & J2  &            &            & 27 Sep 18                 & 2100       & 40–20, 60–40, 80–60, 100–80   \\
				22-25       & M1  & 20         & 69.19      & 28 Sep 18                 & 2135       & 40–20, 60–40, 80–60, 100–80   \\
				26-27       & M1  & 20         & 69.19      & 28 Sep 18                 & 2205       & 50–25, 100–50                 \\
				28-29       & M2  & 20.01      & 69.2       & 29 Sep 18                 & 2035       & 50–25, 100–50                 \\
				30-33       & M2  & 20.01      & 69.2       & 29 Sep 18                 & 2057       & 40–20, 60–40, 80–60, 100–80   \\
				34-37       & U1  &            &            & 5 Oct 18                  & 2000       & 40–20, 60–40, 80–60, 100–80   \\
				38-40       & U1  &            &            & 5 Oct 18                  & 2100       & 50–25, 100–50, 150–100        \\
				41-43       & U2  &            &            & 6 Oct 18                  & 2000       & 50–25, 100–50, 150–100        \\
				44-47       & U2  &            &            & 6 Oct 18                  & 2100       & 40–20, 60–40, 80–60, 100–80   \\
				48-51       & K1  & 9.06       & 75.42      & 8 Oct 18                  & 421        & 40–20, 60–40, 80–60, 100–80   \\
				52-54       & K1  & 9.06       & 75.42      & 8 Oct 18                  & 449        & 50–25, 100–50, 150–100        \\
				55-56       & K2  & 9.04       & 75.4       & 8 Oct 18                  & 2027       & 50–25, 100–50                 \\
				57-60       & K2  & 9.04       & 75.4       & 8 Oct 18                  & 2045       & 40–20, 60–40, 80–60, 100–80   \\
				\midrule
				61-64       & G2  & 15.16      & 72.74      & 16 Oct 19                 & 829        & 50–25, 75–50, 100–75, 150–100 \\
				65-67       & G3  & 15.16      & 72.74      & 16 Oct 19                 & 1812       & 50–25, 75–50, 100–75          \\
				68-70       & K2  & 9.02       & 75.42      & 20 Oct 19                 & 840        & 50–25, 75–50, 100–75          \\
				71-74       & K3  & 9.04       & 75.43      & 20 Oct 19                 & 1934       & 50–25, 75–50, 100–75, 150–100 \\
				75-78       & KK1 &            &            & 22 Oct 19                 & 742        & 50–25, 75–50, 100–75, 150–100 \\
				79-82       & KK2 &            &            & 22 Oct 19                 & 1925       & 50–25, 75–50, 100–75, 150–100 \\
				83-86         & J1  &            &            & 30 Oct 19                 & 324        & 50–25, 75–50, 100–75, 150–100 \\
				87-89         & J2  &            &            & 4 Nov 19                  & 946        & 75–50, 100–75, 150–100        \\
				90-92         & M2  & 19.98      & 69.22      & 29 Nov 19                 & 1434       & 50–25, 75–50, 100–75          \\
				93-96         & M3  & 20.01      & 69.23      & 30 Nov 19                 & 958        & 50–25, 75–50, 100–75, 150–100 \\
				97-100        & O1  & 22.24      & 67.49      & 1 Dec 19                  & 937        & 50–25, 75–50, 100–75, 150–100 \\
				101        & O2  & 22.25      & 67.46      & 1 Dec 19                  & 1957       & 150-100                       \\
				\midrule
				102-105       & G3  & 15.68      & 73.22      & 28 Nov 20                 & 930        & 50–25, 75–50, 100–75, 150–100 \\
				105-108       & G4  & 15.32      & 73.22      & 29 Nov 20                 & 1558       & 50–25, 75–50, 100–75, 150–100 \\
				108-110       & J2  & 17.85      & 71.21      & 30 Nov 20                 & 1458       & 75–50, 100–75, 150–100        \\
				111-114       & J3  & 17.91      & 71.21      & 1 Dec 20                  & 1052       & 50–25, 75–50, 100–75, 150–100 \\
				115-118       & M4  & 20.03      & 69.38      & 2 Dec 20                  & 2016       & 50–25, 75–50, 100–75, 150–100 \\
				119.00        & O2  & 22.41      & 67.8       & 4 Dec 20                  & 953        & 150-100                       \\
				120-123       & O3  & 22.41      & 67.79      & 4 Dec 20                  & 2011       & 50–25, 75–50, 100–75, 150–100 \\
				124-127       & K3  & 9.11       & 75.72      & 12 Dec 20                 & 2335       & 50–25, 75–50, 100–75, 150–100 \\
				128-131       & K4  & 9.06       & 75.74      & 13 Dec 20                 & 1507       & 50–25, 75–50, 100–75, 150–100 \\
				132-134       & KK1 & 7.62       & 77.63      & 14 Dec 20                 & 1226       & 50–25, 75–50                  \\
				135-138       & KK2 & 7.62       & 77.63      & 14 Dec 20                 & 2047       & 50–25, 75–50, 100–75, 150–100 \\
				\midrule
				139-142       & G4  & 15.32      & 73.21      & 3 Mar 22                  & 823        & 50–25, 75–50, 100–75, 150–100 \\
				143-146       & G5  & 15.68      & 73.21      & 4 Mar 22                  & 1030       & 50–25, 75–50, 100–75, 150–100 \\
				147-150       & M5  & 19.99      & 69.23      & 7 Mar 22                  & 957        & 50–25, 75–50, 100–75, 150–100 \\
				151-154       & O3  & 22.24      & 67.5       & 8 Mar 22                  & 806        & 50–25, 75–50, 100–75, 150–100 \\
				155-158       & U3  & 12.5       & 74.04      & 12 Mar 22                 & 1156       & 50–25, 75–50, 100–75, 150–100 \\
				159-160       & K4  & 9.04       & 75.42      & 13 Mar 22                 & 1027       & 50–25, 75–50, 100–75          \\
				161-164       & KK3 & 6.97       & 77.4       & 15 Mar 22                 & 1220       & 50–25, 75–50, 100–75, 150–100
				\\ 
				\bottomrule
			\end{tabular}
		\end{adjustwidth}
	}
\end{table}


\newpage
\begin{figure}[htbp]
	\centering
	\includegraphics[width=0.6\textwidth]{/media/scilab/disk_ranjan/works/backscatter_wc/figures/adcp_moorings_new1.jpg} 
	\captionsetup{justification=justified,font=footnotesize,skip=0.05\baselineskip,width=0.4\textwidth}
	\caption{Map showing region of interest. The slope moorings are deployed at ~ 1000 m depth.}
	\label{fig:fig1}
\end{figure}

\newpage
\begin{figure}[htbp]
	\centering
	\includegraphics[width=0.5\textwidth]{/media/scilab/disk_ranjan/works/backscatter_wc/figures/backscatter_vs_biomass.png} 
	\captionsetup{justification=justified,font=footnotesize,skip=0.05\baselineskip,width=0.7\textwidth}
	\caption{The linear fit line of Biomass (taken in log of biomass) and Backscatter. The linear fit line is within the error range of previous result of \citep{aparna2022seasonal} onto which latest zooplankton volumetric sample data is added.}
	\label{fig:fig2}
\end{figure}


\newpage


\begin{figure}[htbp]
	\centering
	\includegraphics[width=\textwidth]{/media/scilab/disk_ranjan/works/backscatter_wc/figures/biomass_daily_monthly.png} 
	\captionsetup{justification=justified,font=footnotesize,skip=0.05\baselineskip,width=\textwidth}
	\caption{The Daily and monthly averaged biomass for EAS moorings, north (top) to south (bottom). The black contours are marking 215 $mg\ m^{-3}$ biomass.}
	\label{fig:fig3}
\end{figure}

\begin{figure}[htbp]
	\centering
	\includegraphics[width=\textwidth]{/media/scilab/disk_ranjan/works/backscatter_wc/figures/climatology_biomass_ss_chl.png} 
	\captionsetup{justification=justified,font=footnotesize,skip=0.05\baselineskip,width=\textwidth}
	\caption{Monthly climatology of zooplankton biomass is shown in left panels for 7 locations, (bottom is southward). The D215 is shown in solid line. Dashed (dotted) line represents the depth of 23 $^o$C (2.1 $ml\ L^{-1}$ oxygen)contour. The plot labels are as follows: a1 \& a2 for Okha, b1 \& b2 for Mumbai, c1 \& c2 for Jaigarh, d1 \& d2 for Goa, e1 \& e2 for Udupi, f1 \& f2 for Kollam, g1 \& g2 for Kanyakumari. The right panel is zooplankton standing stock and chlorophyll climatology for respective location.}
	\label{fig:fig4}
\end{figure}

\begin{figure}[htbp]
	\centering
	\includegraphics[width=\textwidth]{/media/scilab/disk_ranjan/works/backscatter_wc/figures/aparna_ranjan_climatology_comparison.png} 
	\captionsetup{justification=justified,font=footnotesize,skip=0.05\baselineskip,width=\textwidth}
	\caption{Monthly climatology of zooplankton biomass is shown in left panels for 3 locations which were earlier used in \citep{aparna2022seasonal}; a1, b1 \& c1 is the biomass climatology for Mumbai, Goa and Kollam, d1 is for climatological zooplankton standing stock, e1 is chlorophyll biomass climatology; a2, b2, c2, d2 \& e2 is same but based on data from 2017 to 2023. The D215 is shown in solid line. Dashed (dotted) line represents the depth of 23 $^o$C (2.1 $ml\ L^{-1}$ oxygen) contour. 
	}
	\label{fig:fig5}
\end{figure}


\begin{figure}[htbp]
	\centering
	\includegraphics[width=\textwidth]{/media/scilab/disk_ranjan/works/backscatter_wc/figures/west_coast_wavelet_40m_scale.png} 
	\captionsetup{justification=justified,font=footnotesize,skip=0.05\baselineskip,width=\textwidth}
	\caption{Wavelet power spectra (Morlet) of the 40 m zooplankton biomass plotted against time as abscissa and period in days as ordinate. The wavelet power is in log$_2$ scale, the 95 \% significance is marked in black contours; the cross-shaded region falls under cone of influence.}
	\label{fig:fig6}
\end{figure}

\begin{figure}[htbp]
	\centering
	\includegraphics[width=\textwidth]{/media/scilab/disk_ranjan/works/backscatter_wc/figures/biomass_40m_104m.png} 
	\captionsetup{justification=justified,font=footnotesize,skip=0.05\baselineskip,width=\textwidth}
	\caption{The daily biomass at depth of 40 m and 104 m for all locations shown by grey and cyan curves. The black and blue lines shows the 30 day rolling averaged biomass. }
	\label{fig:fig7}
\end{figure}


\end{document}
