\documentclass{article}
\usepackage{natbib}
\usepackage{multirow}
\usepackage{booktabs}
\usepackage{changepage}
\usepackage{caption} % For caption customization
\usepackage{lineno} % For line numbers
\usepackage{graphicx} % For including graphics
% Add line numbers to the document
\usepackage{geometry}


%\usepackage{hyperref}
\usepackage[colorlinks = true, linkcolor=blue, urlcolor=blue, citecolor=blue]{hyperref}
\usepackage[nameinlink]{cleveref}
\crefname{figure}{Fig.}{Figs.}

\linespread{2} 

{\Large 
	\title{Spatio-temporal variability of zooplankton standing stock in eastern Arabian Sea inferred from ADCP backscatter measurements }}
\author{Ranjan Kumar Sahu, P. Amol, D.V. Desai, S.G. Aparna,  D. Shankar}
\date{\today}
\begin{document}
	

	\maketitle
	\linenumbers
	\section*{Abstract}
	
We use acoustic Doppler current profiler (ADCP) backscatter measurements to map
the spatio-temporal variation of zooplankton standing stock in the eastern Arabian Sea (EAS). The ADCP moorings were deployed at seven locations on the continental slope off the west coast of India; we use data from October 2017 to December 2023. The 153.3 kHz ADCP uses backscatter from sediments or organisms such as copepods, ctenophores, salps and amphipods greater than 1 cm to calculate current profile. The backscatter is obtained from echo intensity using RSSI conversion factor after doing necessary calibrations. The conversion from backscatter to biomass is based on volumetric zooplankton sampling at the respective locations. Analysis of the data over 24--120 m shows that the backscatter and zooplankton biomass decrease from the upper ocean (215 ~mg$^{-3}$ biomass contour) to the lower depths. Changes are observed in the seasonal variation of the monthly climatology of zooplankton standing stock (integral of the biomass over 24--120 m water
column) as we move to poleward along the slope in EAS. The range of variation of standing stock is lowest at Kanyakumari, followed by Okha, which lie at the southern and northern boundary of the EAS, respectively. Complementary variables are used to explain the processes leading to growth or decay of zooplankton biomass.	
	\newpage
	\section{Introduction}
	\subsection{Background}
	Zooplankton plays a vital role in food web of pelagic ecosystem by enabling the hierarchical transport of organic matter from primary producers to higher trophic levels impacting the fish population \citep{ohman2001density} and the carbon pump of the deep ocean \citep{le2016global}. They are presumably the largest migrating organisms in terms of biomass \citep{hays2003review} which occurs in diel vertical migration (DVM). Zooplanktons depend not only on phytoplankton but other environmental parameters (e.g. mixed layer depth, insolation, oxygen, thermocline, nutrient availability, chlorophyll concentration and daily primary production). The biological productivity of the ocean is essentially connected with physics and chemistry \citep{subrahmanyan1959studiespart2, ryther1966primary, qasim1977biological, nair1970primary,banse1995zooplankton,mccreary2009biophysical, vijith2016consequences,amol2020modulation}. The dynamic ocean results in varying physico-chemical properties, leading to bloom and growth of plankton in favourable conditions. The changes are strongly influenced by the seasonal cycle in the North Indian Ocean (NIO; north of ~5 $^o$N of Indian Ocean). The eastern boundary of Arabian Sea contains the West India Coastal Current (\citep[WICC]{ramamirtham1965hydrography, banse1968hydrography, shetye1990hydrography,mccreary1993numerical, amol2014observed, vijith2016consequences, chaudhuri2020observed}) which reverses seasonally, flowing poleward (equatorward) during November to February (June to September). 
	
	A direct consequence of this reversal is the seasonal cycle of thermocline, oxycline and thickness of mixed Layer Depth (MLD) induced by upwelling (downwelling) favourable conditions in summer (winter) eastern Arabian Sea (EAS) facilitated further by wind speed and near-surface stratification. Further, the phytoplankton biomass and chlorophyll concentration changes with the season \citep{subrahmanyan1960studies, banse1968hydrography, levy2007basin, vijith2016consequences}. Upwelling in  summer monsoon leads to maximum chlorophyll growth in the entire EAS \citep{ banse1968hydrography, banse2000geographical, mccreary2009biophysical, hood2017biogeochemical,shi2022phytoplankton}. During winter monsoon, the convective mixing induced winter mixed layer \citep{shetye1992does, madhupratap1996mechanism,mccreary1996four, levy2007basin,  shankar2016inhibition, vijith2016consequences, keerthi2017physical,shi2022phytoplankton} results in winter chlorophyll peak in northern EAS (NEAS) while the downwelling Rossby waves modulate chlorophyll along the southern EAS (SEAS) albeit limited to coast and islands \citep{amol2020modulation}. 
	
	The zooplankton grazing peak is instantaneous with no time delay from peak phytoplankton production \citep{li2000determines,barber2001qn}, but its population growth lags \citep{rehim2012dynamical, almen2020temperature} depending on its gestation period and other limiting aspects. While some studies suggest that the peak timing of zooplankton may not change in parallel with phytoplankton blooms \citep{winder2004climatic}, others indicate that lag exists between primary production and the transfer of energy to higher trophic levels \citep{brock1992interannual, brock1991phytoplankton}. The conventional zooplankton measurements, where only few snapshot/s of the event is captured gives an incoherent or incomplete understanding in terms of spatio-temporal variation of zooplankton \citep{ramamurthy1965studies, piontkovski1995spatial, madhupratap1992zooplankton,madhupratap1996lack,wishner1998mesozooplankton,kidwai2000dd,barber2001qn,khandagale2022seasonal} as much information is revealed by later studies \citep{jyothibabu2010re, vijith2016consequences, shankar2019role, aparna2022seasonal} using high resolution data. Calibrated acoustic instruments such as acoustic Doppler current profiler (ADCP) along with relevant data can be utilised to understand small scale variability \citep{nair1999arabian, edvardsen2003assessing, smith2005mesozooplankton, smeti2015spatial, kang2024acoustic}, the complex interplay between the physico-chemical parameters and ecosystem \citep{jiang2007temporal, potiris2018acoustic, shankar2019role, aparna2022seasonal, nie2023influence}, the zooplankton migration \citep{inoue2016diel,ursella2018evidence, ursella2021diel} and their seasonal to annual variation \citep{jiang2007temporal, hobbs2021marine,liu2022seasonal, aparna2022seasonal}.
	
    The relationship between backscatter and the abundance \& size of zooplankton was described by \citet{greenlaw1979acoustical} wherein it was pointed out that single frequency backscatter can be used to estimate abundance if mean zooplankton size is known. A drastic increase in study temporal and spatial variation of zooplankton biomass using  backscatter-proxy came in 1990s by introduction of high frequency echo sounders, with studies \citep{flagg1989use, wiebe1990sound, batchelder00981, greene1998three, rippeth1998diur} methodically showing acoustic backscatter estimated zooplankton biomass.
	Net sampling augmented ADCP backscatter have been used to study DVM and the spatial and temporal variability of zooplankton biomass in different marine regions, such as the Southwestern Pacific, the Lazarev Sea in Antarctica and the Corsica Channel in the north-western Mediterranean Sea \citep{cisewski2010seasonal,hamilton2013links, smeti2015spatial, guerra2019zooplankton}. The first such study to exploit the potential of ADCPs in EAS was carried out by \citet{aparna2022seasonal} (A22 from hereon) using ADCP moorings deployed on continental slopes off the Indian west coast.	In their work, they showed that the zooplankton standing stock (ZSS) in fact declines during upwelling facilitated increase in phytoplankton biomass. The unusual interaction implies the break down of existing understanding of predator-prey relationship in fundamental level of marine food chain.
	
	\subsection{Objective and scope of the manuscript}
	
	A network of ADCPs has been installed off the continental slope and shelf on the west coast of India. This ADCPs have enabled a rigorous view of intraseasonal to seasonal scale variability \citep{amol2014observed, chaudhuri2020observed} of WICC. In the recent study A22 have used ADCP moorings off  Mumbai, Goa and Kollam to explain the temporal variability of zooplankton biomass. The study showed that the zooplankton peaks (and troughs) is not only non-uniform in latitude but also heavily influenced by the oxygen minimum zone, MLD and the seasonal upwelling/downwelling conditions. Stark contrast in the phytoplankton bloom and subsequence  growth of zooplankton or the lack thereof was observed in the EAS regimes.
	
    We extend the work of A22 by presenting data from four additional moorings in the EAS, showcasing the deviations of seasonal cycle from climatology, and discussing the significant intraseasonal variability of biomass and standing stock revealed by the ADCP data. The paper is organized as follows; datasets and methods employed are described in section 2. Section 3 describes the observed climatology of zooplankton biomass and standing stock. A comparison is drawn to the results of A22. Further, the seasonal cycle of zooplankton biomass and standing stock is discussed with relation to the MLD, oxygen, temperature and circulation in determining the biomass is discussed in results section 4. Section 5 delves deeper into the intraseasonal variability with summary and conclusion in section 6.
	
	\section{Data and methods}
	The  backscatter data from ADCP and the zooplankton samples collected from the periphery of mooring is described in this section. The backscatter derived from the echo intensity of the seven ADCP mooring deployed on the continental slope off the Indian west coast is the primary data we have use in this manuscript. The moorings details are summarized in \autoref{tab:table1}. In situ biomass data from volumetric zooplankton samples are used to validate and correlate with backscatter. The chlorophyll data is obtained from \href{https://data.marine.copernicus.eu/products}{marine.copernicus.eu}. In addition, we have used the monthly climatology of temperature and salinity \citep{chatterjee2012new} and the net primary productivity from MODIS (Moderate Resolution Imaging Spectroradiometer) and VIIRS (Visible Infrared Imaging Radiometer Suite) from global NPP estimates (\href{http://sites.science.oregonstate.edu/ocean.productivity}{http://sites.science.oregonstate.edu/ocean.productivity}). 
	
	\subsection{ADCP data and backscatter estimation}
    The ADCPs were deployed on the continental slope off the Indian west coast (\cref{fig:map}), off Mumbai, Goa, Kollam and Kanyakumari, and later extended to three more sites to cover the entire EAS basin from Okha (22.26$^o$N) in north to Kanyakumari (6.96 $^o$N) in south. The other two ADCPs are  Jaigarh at central EAS (CEAS) and Udupi (primarily at SEAS regime) in the transition zone between CEAS \& SEAS. The extended moorings were deployed in October 2017, though Kanyakumari had been deployed earlier too. However, only Mumbai, Goa and Kollam were part of the previous backscatter study by A22. The moorings are serviced on yearly basis usually during October--November or sometime during September--December (depending on ship availability). The ADCPs are of RD Instruments make, upward-looking and operate at 153.3 kHz. While utmost care is taken to position the instrument at  $\sim$ 200 m depth, yet for some deployments it's shallow or deeper owing to drift caused by floater buoyancy-anchor weight balance. Data was collected at hourly interval and the bin size was set to 4 m. The echoes at surface to 10\% range ($\sim$ 20 m) means the data at these depths is rendered useless and is discarded from further use.  We have followed the methodology laid down in A22 to derive the backscatter time series from ADCP echo intensity data. The gaps up to two days are filled using the grafting method of \citet{mukhopadhyay2017st} once the zooplankton biomass time series is constructed.
    
	\subsection{Zooplankton data and estimation of biomass}
	The zooplankton  samples were collected in the vicinity ($\sim$ 10 km) of ADCP mooring site twice, once prior retrieval and again post deployment of moorings so that there is overlap in the ADCP time instance and in situ zooplankton samples. The sampling is done at the mooring location during servicing cruises on board RV Sindhu Sankalp and RV Sindhu Sadhana (\cref{tab:table2}). Multi-plankton net (MPN) (100 $\mu m$ mesh size, 0.5 $m^2$ mouth area) was used to get samples in the pre-determined depth ranges; water volume filtered was calculated by the product of sampling depth range and the mouth area of net. The depth range and timing of sample collection was different throughout the MPN hauls (refer \autoref{tab:table2}). From 2020 onward, the depth-range was standardized to the bins of 0 -- 25, 25 -- 50, 50 -- 75, 75 -- 100, 100 -- 150 (units are in meters). The backscatter obtained earlier is averaged in vertical corresponding to the specific MPN hauls for each site. The backscatter is linear regressed with respective biomass to establish their relationship, which has been demonstrated in numerous previous studies \citep{flagg1989use,heywood1991estimation,jiang2007temporal,aparna2022seasonal}.
	
	\subsubsection{Biomass time series and estimation of standing stock}
	
	The zooplankton biomass time series (\cref{fig:dailynmonthly}) is created from the above derived linear relationship. The standing stock is determined by taking the depth integral of biomass over the water column. To maintain the consistency of standing stock estimation, only those deployments that doesn't lack data at any depth in the entire range of 24--120 m are considered for analysis as in A22. The lack of data in the above mentioned depth range is due to deviation in positioning of ADCP sensor in the water column. A swift alteration in bathymetry along the continental slope implies that the mooring might anchor at a different depth than planned, hence a change in the predicted position of ADCP. This leads to gap in data at few mooring sites for some year. For example, for the northern-most mooring at Okha, data is not available for the entire upper 120 m depth for the second deployment. Also at Jaigarh, where the surface to $\sim$60m data (in 3rd deployment) and Kollam, where 80 m and below (in 4th deployment) is unavailable and hence discarded from standing stock estimation. There are few deployments where no data or bad data was recorded e.g, at Udupi (4th deployment) and Kanyakumari (6th deployment).  	
	
	\subsection{Mixed-layer depth, temperature and oxygen}
	As we are using a 153.3 kHz ADCP moored at $\sim$ 150 m, the top $\sim$ 10\% of data is unusable because of surface echoes. MLD in EAS is of the order $\sim$ 20 to 40 m during summer monsoon \citep{shetye1990hydrography,shankar2005hydrography,sreenivas2008monthly} especially in the SEAS \citep{shenoi2005hydrography}, but during winter the MLD in northern NEAS remains deep \citep{shankar2016inhibition}. Although it is possible to use the near-surface ADCP data after due noise correction, it is beyond the scope of present study. The temperature data is used from \citet{chatterjee2012new}, a monthly climatology having 1$^o$ spatial resolution. Monthly climatology of oxygen data is obtained from World Ocean Atlas 2013 \citep{garcia2013oxygen} which contains objectively analyzed 1$^o$ climatological fields of in situ measurements. 
	
	\subsection{Chlorophyll and net primary productivity data}
	Previous study based on ADCP data of EAS A22 have used SeaWIFS based chlorophyll data for comparison with climatology of ZSS. The SeaWIFS was at its end of service in 2010, hence we use new chlorophyll product. While the chlorophyll product only captures chlorophyll at surface, the productivity models consists information in depth containing the subsurface chlorophyll maxima. The present study has been conducted using Global Ocean Colour, biogeochemical, L3 data obtained from the  \href{https://doi.org/10.48670/moi-00280}{E.U. Copernicus Marine Service Information}. The daily data is available at a spatial resolution of 4 km. 

	%While chlorophyll is used to compare with the variation in climatology of zooplankton standing; the growth efficiencies of zooplankton are directly linked to primary production levels, emphasizing the interconnectedness between primary producers and consumers in marine food webs \citep{friedland2012pathways}. In their study, A22 has emphasized on the collapse of the predator-prey relationship between zooplankton-phytoplankton using climatological data. We showcase their interdependency or the lack thereof using net primary productivity models.
	%Moderate Resolution Imaging Spectroradiometer (MODIS) based net primary productivity (NPP) data at a resolution of 0.16$^o$ x 0.16$^o$ was obtained from Oregon State University. They have employed three different schemes to obtain NPP from Chlorophyll concentration. Those are discussed below in brief. The first is Vertically Generalized Production Model (VGPM). The NPP (a rate term) is to be derived from chlorophyll (a standing stock) using chlorophyll-specific assimilation efficiency for carbon fixation. The single biggest unknown in all models based on chlorophyll is how this rate term is described. VGPM considers the primary productivity to be dependent on day length and maximum daily NPP within a water column. The second is Carbon-based Productivity Model (CbPM) which NPP to phytoplankton carbon biomass and growth rate. The third is Carbon, Absorption, and Fluorescence Euphotic-resolving (CAFE) mode; first described in \citet{silsbe2016cafe} takes various other factors into NPP calculations. We explore these NPP models and try to explain the variation in ZSS.
	
	\section{Climatology of zooplankton biomass and standing stock}
	The high and low productive biomass regime in upper and deeper depths is demarcated using a biomass contour as in A22 who chose 215 mg~m$^{-3}$ as such, and it's depth is labeled as D215. However, in the present scenario we've moorings deployed at farther ends of EAS, namely Kanyakumari (at SEAS) and Okha (at NEAS). The choice of biomass contour isn't abrupt; first, it is carefully chosen to accommodate the seasonal variation, as a shift to biomass contour lower than the z215 would be unviable as our data is only till 140 m depth. For example in the case of Kollam, the D215 exceeds 140 during few months of 2022 (\cref{fig:dailynmonthly}). A higher biomass contour would lead to subdued view of the seasonal cycle as in the case of Kanyakumari and Okha where 215 mg~m$^{-3}$ biomass contour is often low enough to reach $\sim$20--30 m depths (\cref{fig:dailynmonthly}), hence z175 is chosen here. Second, it allows us to link the seasonal variation of biomass to the physico-chemical properties. Climatology of zooplankton biomass and ZSS is discussed at locations northward starting from southernmost mooring site i.e, Kanyakumari. The time series is discussed briefly in the following subsection.
	The monthly climatology of biomass and ZSS is computed for all locations having valid data in 24--140 m depth range (\cref{fig:zsschlclim}). A comparison is made in later paragraphs with availability of new data.
 
	 
	\subsection{Southern EAS}
	During mid-March off Kanyakumari, the depth of 23 $^o$ C isotherm (henceforth D23) shallows along-with oxycline (marked by 2.1 ~ml~L$^{-1}$, a higher oxygen contour as it lies outside OMZ core) and a rise in biomass is observed (\cref{fig:zsschlclim} g1). The z175 is shallower from May onward to October and the zooplankton biomass is comparatively higher than rest of the year. D175 deepens starting from October and the relatively high biomass in water column is maintained till late December. However, the deepening of D175 isn't reflected as an increase in ZSS because of low biomass in the entire water column. A gradual increase is seen in the chlorophyll biomass starting from April and the peak is attained in June (\cref{fig:zsschlclim} g2). The ZSS is increased in June, however the growth is minimal. There is almost no seasonal variation in ZSS off Kanyakumari (ZSS std, 0.67 gm~m$^{-2}$) as compared to the ZSS variation at the nearest northern mooring site off Kollam (ZSS std, 1.25 gm~m$^{-2}$), where a strong seasonal cycle is observed and the D215 is deeper for any given month.
	
	Off Kollam, a higher biomass is present in the larger portion of water column and the D215 is at $\sim$ 110 m during Mar--May (\cref{fig:zsschlclim} f1). Starting from March, the D215 begins to shallow with progress in time till August. During this period, a sharp decrease is seen in the D23 ($\sim$ 80 m in June to September) while the oxycline (1.7~ml~L$^{-1}$ overshoots and reaches $\sim$ 40 m (\cref{fig:zsschlclim} f1). A steep rise in chlorophyll biomass is seen off Kollam and its peak is attained in August (\cref{fig:zsschlclim} f2). The ZSS declines in the same period and reaches a minimum when the chlorophyll biomass is at its peak. The chlorophyll biomass decreases rapidly in the following months, while the ZSS increases and a maximum is seen during October. This feature was previously reported by A22, highlighting an imbalance in the interaction between zooplankton and phytoplankton.
	
	A similar feature is seen further north, off Udupi which sits at the transition zone of SEAS \& CEAS, albeit with a relatively weaker zooplankton biomass. The peak of chlorophyll and minimum of ZSS occurs in September (\cref{fig:zsschlclim} e2) which is one month later than off Kollam. The 2.1 ~ml~L$^{-1}$  oxygen contour overshoots thermocline, however it reaches to a much shallow depth of $\sim$ 20 m during July to October unlike any other location in our EAS study area. The D215 vaguely follows D23; with the gradual shallowing from March onward reaching $\sim$ 60 m in September and a steep decline afterwards till November (\cref{fig:zsschlclim} e1).
	% The decrease of biomass with depth is moderate in comparison to Kollam.
	%put in discussion. 
	%The seasonal cycle of the thermocline, oxycline, and D215 follow the same pattern.  What changes is the range over which these three curves move during the year. 

	\subsection{Central EAS}
	Off Goa, the D215 seasonal trend is as in A22 and is similar to Udupi since it is entirely restricted by D23 and 1.7 $ml \ L^{-1}$ oxygen contour that closely follows it. 
	% During March-May, the D215 is at $\sim$ 80 -- 100 m which shallows with onset of summer monsoon (\cref{fig:zsschlclim} d1); the chlorophyll biomass increases during this period and the maximum occurs in August after which the chlorophyll biomass and ZSS (\cref{fig:zsschlclimcomp}) both decrease in September.
	Although we witness an increase in chlorophyll biomass in October, the D215 is restricted to the $\sim$ 50 m in this period  and the ZSS is at minimum  similar to off Udupi (Kollam) during September (August). The ZSS rapidly increases and reaches its maximum in January, sustained till March and then gradually declines. Unlike the previous locations, the biomass off Goa decreases rapidly below the z215 as reported earlier in A22, reaching as low as 60 mg~m$^{-3}$ at 130 m during June to September (\cref{fig:zsschlclim} d1).
	 
	The ZSS off Jaigarh is identical but stronger to that of off Goa, owing to higher biomass above z215 and the comparatively deeper D215 (\cref{fig:zsschlclim} c1). 
	%The D215 follows D23 \& oxycline for most of the year and it only exceeds during October-December. 
	From the ZSS maximum in February (\cref{fig:zsschlclim} c2), it steadily decreases and attains a minimum in September, a rapid rise is seen in the following months. What's intriguing is a presence of strong seasonal cycle in ZSS off Jaigarh (std 3.24 gm~m$^{-2}$, highest among all locations) although the seasonal variation in chlorophyll biomass (\cref{fig:zsschlclim} c2) is visibly non-existent (0.05 mg~m$^{-3}$ Chl std, lowest among all locations). This is an exact opposite scenario of Kanyakumari, where an insignificant seasonal variation in ZSS (0.67 gm~m$^{-2}$ ZSS std) is seen even though the chlorophyll biomass varies strongly (0.51 mg~m$^{-3}$ Chl std). 
		
	Starting from Kollam (\cref{fig:zsschlclim} f1) and moving northward to Jaigarh (\cref{fig:zsschlclim} c1), we see that the core of high zooplankton biomass gradually shifts from summer (off Kollam) to winter monsoon (off Jaigarh), with the transition of upper ocean zooplankton biomass happening along Udupi and Goa. On the contrary, the chlorophyll biomass tends to have low seasonal range as we move northward from SEAS, with Jaigarh having the least seasonal variation. This shift along with winter monsoon facilitated deeper thermocline leads to an even larger impact on ZSS.
	 
	\subsection{Northern EAS}
	Further north off Mumbai the D215 is deeper in December to early April, resulting in a higher ZSS (\cref{fig:zsschlclim} b2). D23 follows D215 and the oxycline follows an erratic pattern, reaching depths $>$ 140 during January to March (\cref{fig:zsschlclim} b1); when a higher biomass is observed above z215. The chlorophyll biomass shows seasonal variation albeit lower than the SEAS counterpart. The ZSS increases rapidly from its minima in October in the following month as the D215 deepens and the maximum occurs in February. The chlorophyll biomass decreases from March and a gradual decrease in ZSS is seen till July, after which the ZSS basically flattens even though the chlorophyll increases. 
	
	At the northernmost site of EAS i.e, off Okha, a noticeable feature is a much higher oxygen in upper ocean except during summer monsoon, therefore  a higher oxygen value is used for the oxycline contour. The biomass above z175 is much weaker (\cref{fig:zsschlclim} a1) leading to a relatively lower ZSS (\cref{fig:zsschlclim} a2) compared to Mumbai. The D175 shallows from February to it's minimum in August. There's two chlorophyll peak off Okha, one in February due to convective mixing induced deepening of MLD \citep{wiggert2005monsoon,levy2007basin,keerthi2017physical,shankar2016inhibition} and the other during August in summer monsoon \citep{wiggert2005monsoon,levy2007basin}. The ZSS remains flat in summer monsoon period i.e, June to September, although the chlorophyll biomass increases in this time. Afterwards, ZSS gradually increases and attains its maximum in February same as the chlorophyll biomass. ZSS sustains this maximum till March, declines rapidly in April and then gradually till July.
	 
	\subsection{Comparison to biomass and ZSS climatology of A22}	 
	It is observed that D215 is shallower at all locations and as a result a lower biomass and ZSS as seen in the climatology of the present study (\cref{fig:zsschlclimcomp}). The difference in D215 is prominent off Goa; while in the previous climatology (\cref{fig:zsschlclimcomp} b1) the D215 is deeper and lies along D23, in the present climatological data the D215 is shallower and lies $\sim$ 20--40 m above the D23 during January to April  (\cref{fig:zsschlclimcomp} b2). A relatively lower biomass is present above z215 year round which reflects in overall lower ZSS of Goa and Mumbai. In the present data, the ZSS maximum off Mumbai occurs in March instead of February (A22), due to a lower ZSS value. The second maximum occurs in August (\cref{fig:zsschlclimcomp} d1) and is less pronounced in recent data (\cref{fig:zsschlclimcomp} d2). There is dramatic decrease in the minima off Mumbai that occurs in October and ZSS increases rapidly afterwards till February. Off Kollam, higher biomass occurs from May to June in A22, and from May to June and September to November in the present study, with a ZSS minima in August (\cref{fig:zsschlclim} c2, d2). The higher ZSS on either side to this minima is less pronounced in A22. This difference in ZSS is clearly seen in the correlation, which is 0.60 off Kollam, while it is 0.94 and 0.98 off Mumbai and Goa, respectively. The correlation reflects ZSS trend similarity, not magnitude deviation over time. In the present study, chlorophyll biomass peaks across all locations in August, revealing a zooplankton-phytoplankton relationship discrepancy off Kollam, consistent with A22 findings.
	
	\subsection{Time series of zooplankton biomass}
	A preliminary analysis of the biomass time series in daily and monthly averaged scale shows that the biomass decreases with increasing depth (\cref{fig:dailynmonthly}) at all the seven locations. Rate of biomass decrease with depth, roughly defined as the difference between mean biomass at 40 m  and 104 m depth is highest off Jaigarh and Mumbai as it has higher biomass in upper ocean (\cref{fig:compfourty} c,b) and lowest off Kanyakumari. This is followed by CEAS locations Goa and Udupi (\cref{fig:compfourty} d,e). While the biomass decrease with depth is lower off Kollam from 2017 to 2020, it becomes considerably high from thereon (\cref{fig:compfourty} f). 
	%Moving poleward along the slope, mean biomass at 40 m off Kanyakumari is the least $\sim$ 207 mg~m$^{-3}$ which increases drastically to 272 mg~m$^{-3}$ off Kollam. It decreases till Goa and then increases to a maximum of 278 mg~m$^{-3}$ off Jaigarh. Off Mumbai the mean biomass is 272 mg~m$^{-3}$, and further north off Okha, it declines to 230 mg~m$^{-3}$. A similar trend is observed in mean biomass at 104 m depth of all locations and their corresponding standard deviation. 
	A pattern that develops with lower mean biomass off Okha and off Goa bifurcated by higher mean biomass off Mumbai \& Jaigarh; while the lower mean biomass off Udupi and off Kanyakumari is divided by higher mean biomass off Kollam. From standard deviation of biomass it is inferred that the sites with higher biomass tends to have higher variation over time as in the case of Mumbai, Jaigarh and Kollam. The mean, standard deviation of biomass, ZSS and chlorophyll are shown in \autoref{tab:table3}. Variation in the monthly average or seasonal cycle over time suggests significant interannual variability.	 A comparatively weaker decline in zooplankton biomass with respect to depth off Okha (\cref{fig:dailynmonthly} a1,a2) at NEAS is agreeing with earlier reported data \citep{wishner1998mesozooplankton, madhupratap2001mesozooplankton, smith2005mesozooplankton,jyothibabu2010re}. The sites at SEAS, especially off Kanyakumari and 2017 to 2020 off Kollam also have weaker decrease \citep{madhupratap2001mesozooplankton, jyothibabu2010re, aparna2022seasonal}. However, post 2020 the decline in biomass with depth off Kollam is high owing to a strong bloom in these years reflected as D215 deepening. During winter monsoon, D175 and D215 is deep throughout EAS, but the occurrence of high biomass is distinct to each regime of EAS. Upper ocean shows considerably high biomass during winter at NEAS, on the contrary at SEAS the upper ocean shows higher biomass during summer monsoon even though the D215 \& D175 is shallower during this period. 
	 
	\section{Seasonal cycle and variability}
	\label{sec:seasonalcyclezss}
	This section will deal with a discussion on the seasonal cycle and variability of biomass and ZSS in annual and intra-annual scale along the three regimes of EAS. 

	The time series data is incomplete due to instrument issues or incorrect ADCP placement. For example, Okha's second deployment (2019--2020) lacked data for the top 140 meters because the instrument was too deep. This makes it hard to analyze annual cycles in regions with limited data. Therefore, we consider locations other than Okha and Jaigarh for the 40 m biomass in annual scale (\cref{fig:wave40104}). The ZSS time series is obtained by integrating the biomass over 24--120 m of water column, (\cref{fig:wavess}). 
	%%% IMPORTANT In NEAS (SEAS) regime, the ZSS maximum occurs during January to late march and early April (September to November). 
	The biomass is decomposed into distinct period bands spanning days to months. Among these, DVM is the simplest variation, determining zooplankton biomass at a given depth with higher (lower) biomass at night (day). On a longer time scale, annual variability reflects changes over the course of a year often influenced by seasonal cycles like monsoons. Intra-annual variability captures fluctuations that occur between seasons, shorter than a year but longer than a season. Intraseasonal variability is about shifts occurring within a season, typically lasting weeks to months and driven by short-term environmental changes. The strength and contribution of components of variability changes over time  and differs between EAS regimes. For instance, in 2019 off Kollam (\cref{fig:variability}), intraseasonal variability was dominated by period $<$ $\sim$ 30 days. An increase in biomass during the summer monsoon was due to low-frequency variabilities. However, a sharp decline in August resulted from reduced intra-annual and intraseasonal variability, although a weakly positive annual variability was present.  
	
	From the linear equation correlating biomass and backscatter, the upper and lower bound of error limits equals to $\sim$  14 $ mg\ m^{-3}$ (\cref{fig:bstobm}). The standard deviation incorporating 99.73 \% data i.e, $\pm$ 3 * $\sigma$ of intraseasonal variability is $\pm$ 40 mg~m$^{-3}$ resulting in its range of 80 mg~m$^{-3}$. The intra-annual (annual) variability also has a range of 80 mg~m$^{-3}$ (35 mg~m$^{-3}$). This higher range of variability compared to the error range permits us to infer information reliably. 
	
	\subsection{SEAS}
    The annual cycle of biomass off Kanyakumari (Udupi \& Kollam) is weak (strong), but it varies in time, for example, off Kollam, the wavelet power is stronger post 2020 (\cref{fig:wave40104} f). The absence (presence) of ZSS annual cycle off Kanyakumari (off Udupi) is confirmed with wavelet analysis (\cref{fig:wavess} g1). A biennial peak is observed in Fourier analysis of ZSS of Kollam agreeing with A22.  Along with the annual cycle, we observe presence of semi-annual ($\sim$ 180 days) cycle at most locations and together they constitute the seasonal cycle which weakens with depth at SEAS sites.

	To capture the annual variability, the biomass is filtered with Lanczos filter within period of 300 to 400 days (\cref{fig:annual}). The annual variability off Kanyakumari (22 $mg / m^{-3}$) is least among all mooring sites, but Kollam and Udupi show strong annual variability. The variability at intra-annual (100--250 days) band tends to be stronger compared to the annual variability and is strengthened during late summer monsoon to transitional monsoon period (\cref{fig:intraannual}) as seen during 2021 (off Udupi) and 2021,2022 (off Kollam). Fourier analysis of the daily biomass time series suggests presence of signals within the intra-annual band off Kanyakumari, e.g, power peaks at $\sim$ 140 and $\sim$ 220 days implying that the variability is strictly restricted to narrow bands within intra-annual band. 
	
	\subsection{CEAS}
	Annual cycle of biomass is comparatively weak off Goa (CEAS) contrary to results of A22 which could be due to shorter time record and low biomass in the recent years as reflected in its ZSS for 2018 to early 2020 (\cref{fig:wavess} d1). At Goa and Jaigarh, we see that the semi-annual period dominates seasonal cycle in the same duration (\cref{fig:wave40104} d). The dominance of semi-annual cycle is also seen off Kollam. Although the dominance of semi-annual period in seasonal cycle is seen only for some years, a similar feature was discussed for WICC where the intra-annual component dominates the seasonal cycle as we go equatorward with a change in the strength of intra-annual component in time \citep{chaudhuri2020observed}. The higher intra-annual variability often coincides with shallow D215/D175, for example off Goa during 2020 and 2022, and weak variability coincides with constant depth of D215. This transient nature of variability is due to the spread of energy among all intra-annual periods for 2022 (\cref{fig:wave40104}) off Goa, while during 2020 the wavelet energy is only present in the semi-annual periods, resulting in a overall weaker intra-annual component.
	
	Upward phase propagation in filtered zonal and meridional currents is noted at most mooring sites \citep{amol2014observed} and in the annual filtered biomass off Goa and southern moorings, we shed a light on this in discussion section using wavelet coherence result.
	
	
	\subsection{NEAS}
	Off Mumbai, a strong annual cycle ($\sim$ 365 days) dominates the seasonal cycle throughout the time series of biomass (40 m \& 104 m \cref{fig:wave40104}) and ZSS (\cref{fig:wavess}). The semi-annual cycle is present at 40 m (\cref{fig:wave40104}) off Mumbai but weakens at 104 m(\cref{fig:wave40104}) resulting in a annual cycle dominated ZSS. Analysis reveals the presence of strong semi-annual cycle off Okha (at 104 m only). 
	
	The annual variability is strong off Mumbai and Jaigarh (at CEAS \& NEAS transition zone) with 3 $\sigma$ range of 41 and
	54 mg~m$^{-3}$, respectively, which is highest among EAS regimes.  As observed in ocean currents \citep{amol2014observed, chaudhuri2020observed}, the annual filtered biomass decreases moderately with depth off Mumbai and the three CEAS sites than off Kollam (\cref{fig:annual}). Intra-annual variability of biomass decreases poleward (excluding Kanyakumari) with higher variation seen off Kollam, Udupi and Goa. Off Okha the annual (intra-annual) variability is weaker (comparable) to the variability off Mumbai but with a weak seasonal cycle at upper ocean.
	
	
	\section{Intraseasonal variability}
	The intraseasonal band, defined as the variability occurring between periods of few days to 90 days is split into two categories; a high-frequency (period $<$ 30 days) and a low-frequency (30 $<$ period $<$ 90 days) component. The presence of significant variation in the 30-day running mean with recurring bursts are seen in the daily data and in the  wavelet analysis of biomass at 40 m and 104 m (\cref{fig:wave40104}) as bursts during few months distinctive to each mooring location.  

    The wavelet power at 40 m in low-frequency intraseasonal band peaks during September to December off Mumbai, Jaigarh and Kanyakumari while no such general observation is found in other locations. However, the wavelet power at 104 m in comparison to 40 m suggests a decrease in its strength at respective locations. Lanczos filtered biomass in 30 to 90 day period shows that the intraseasonal variability is strong during August to November off all location (\cref{fig:intraseasonal}) and is often coherent along much of the EAS slope as seen during 2018. This is in contrast to the WICC intraseasonal band which is strong during winter monsoon at slope \citep{amol2014observed, chaudhuri2020observed} and shelf \citep{chaudhuri2021observed}. The low frequency intraseasonal variability is higher in the upper ocean often limited to the upper 50--70 m, however it can extend to deeper depths ($\sim$ 140 m) for some years e.g, off Jaigarh and Goa during September--November 2018 (\cref{fig:intraseasonal} f). The magnitude of low-frequency component of intraseasonal variability is high as we move equatorward till Kollam much like the intraseasonal currents \citep{amol2014observed,chaudhuri2020observed,chaudhuri2021observed} and declines off Kanyakumari (\cref{fig:intraseasonal} g). 
    
    The intra-seasonal component is transient in nature and its magnitude is higher than the low frequency variabilities. A strong dependency of zooplankton biomass on the intra-seasonal variation has implication on the sampling of zooplankton using cruises.     A servicing cruise along the EAS moorings takes about 12 to 15 days excluding the time to and fro from port to first/last mooring \citep{ chaudhuri2020observed, aparna2022seasonal}. However, a sampling cruise dedicated to study the spatial variation of zooplankton \citep{madhupratap1992zooplankton,smith1998seasonal,wishner1998mesozooplankton, kidwai2000dd}, say for summer monsoon may last a month or more with coarse sampling interval and hence fail to capture the actual biomass within a season for a fair spatial comparison. One such occasion is a dip in zooplankton biomass off Kollam because of intraseasonal variability during August, 2019 (\cref{fig:variability}). The resulting biomass is low even though the primary production in SEAS \citep{ashadevi20101070, jyothibabu2010re} is high and subsequent zooplankton biomass is supposed to be high. 
    
    The species distribution of phytoplanktons and further zooplanktons in EAS is determined by intricate play based on predation, environment, competition \citep{raghukumar2003marine} and hence the size distribution changes \citep{madhupratap1996lack,kidwai2000dd,raghukumar2003marine, smith2005mesozooplankton,}, with few species dominating in certain seasons. Habitat patchiness, i.e, irregular distribution of habitats and resources in the deep-sea environment \citep{eggleston1998organism,raghukumar2003marine} contributes to high biodiversity which in turn can have implications for ecosystem dynamics. A high intraseasonal variability in zooplankton biomass suggests that patchiness in the deep-sea environment isn't solely driven by seasonal cycles but also occurs within individual seasons. Further, it can be theorized that patches-specific species will tend to dominate a duration during episodes of intraseasonal bursts which can be verified with carefully planned in situ zooplankton observations.
    
    It is to be noted that while the annual cycle dominates the WICC \citep{amol2014observed, chaudhuri2020observed,chaudhuri2021observed}, contrary to this, the intraseasonal variability that dominates the zooplankton biomass along EAS. On the other hand, the direction of WICC at any given time of the year is unpredictable \citep{chaudhuri2020observed} and advection \& entrainment can affect zooplankton biomass, we can expect it to produce such behaviors that can only be resolved with high-frequency intraseasonal variations. Strong peaks in intraseasonal band in chlorophyll was evident in Lombscargle periodogram (figure not shown), similar to zooplankton biomass and ZSS, but lacked concrete evidence of direct correlation. 
	
	\section{Discussion}

	\subsection{Summary}
	The zooplankton biomass and standing stock across different regions of EAS was examined in this article, highlighting their spatio-temporal trends in the light of physico-chemical parameters using the multi-yearlong ADCP backscatter data from 2017 to 2023. 
	
	The findings shows notable seasonal variation in zooplankton biomass and ZSS; In SEAS the higher biomass is observed during summer monsoon, while in NEAS the high biomass is observed during winter monsoon with transition of peak biomass happening gradually along CEAS regime (\cref{fig:zsschlclim}). Off Kollam, a unique double peak in ZSS occurs, one during May to July and another in September to November, suggesting a complex interplay between environmental drivers and zooplankton growth (\cref{fig:zsschlclim} f2). Off Kanyakumari, the seasonal variation in ZSS is non-existent even though a dramatic seasonality is seen in primary production. Climatology shows strong decline in biomass w.r.t. depth off Goa, then NEAS sites off Jaigarh, Mumbai and Okha followed by SEAS locations off Udupi, Kollam and Kanyakumari.

    % edit start 	
	Seasonal cycle play a crucial role in regulating biomass variability. A strong annual cycle is observed in Northern sites like Mumbai and Jaigarh (\cref{fig:wave40104}), with biomass peaking during winter monsoon months (\cref{fig:zsschlclim}). However, the Southern and Central regions, particularly off Kollam, exhibit more complex patterns. Off Kollam, the presence of a weak annual cycle and a stronger semi-annual cycle is noted along with a moderately strong biennial cycle. The semi-annual cycle is especially prominent in the Southern EAS, where it contributes significantly to the seasonal biomass changes, while Northern regions is dominated by annual cycle. 
	
	Intraseasonal variability, particularly in the 30--90 day range, is found to influence zooplankton biomass significantly, especially in the summer monsoon months (\cref{fig:intraseasonal}), while the high frequency (period $<$ $\sim$ 30 days) variability determine changes in smaller temporal scale (\cref{fig:variability}). Intraseasonal variability is higher in the Southern EAS, with the Northern regions displaying more stable patterns. The variability in annual scale is weak, while that in intra-annual scale is often comparable to intraseasonal variability. The dependence of biomass on the current is investigated showing linkage in the annual scale as seen off Goa at near-surface depths. 
	
	% edit end 
		
	\subsection{Physico-chemical drivers of zooplankton biomass}
	Numerous factors have an impact on the zooplankton population dynamics and growth in the EAS. Throughout the summer monsoon, the Somali current, which flows clockwise in Arabian sea, is essential in moving oxygen-depleted waters creating a perennial oxygen minimum zone (OMZ) \citep{sarma2020potential,sudheesh2022omz}. The net transport of water in upper 500 m of northern Arabian sea is about 5 Sv and a majority of the replaced waters comes from upwelling in the eastern Arabian sea \citep{shi1999remotely} during summer monsoon with the high-nutrient water covering $\sim$ 500--700 km from coast \citep{morrison1998seasonal}. Upwelling supplies nutrients to the surface \citep{Kumar.2000}, but it also plays a role in the creation of hypoxic conditions, which can restrict the kinds of zooplankton species that can survive in these waters \citep{jayakumar.2004}. The upwelling starts in early by February itself off SEAS, but it occurs much later during May farther north along the coast \citep{banse1968hydrography,Kumar.2000,vijith2016consequences,sarma2020potential} albeit weaker than the southern counterpart. The deepening of MLD in winter due to convective mixing during \citep{marra2005jgofs, shankar2016inhibition,shi2022phytoplankton} leads to dilution of zooplankton grazers in water column \citep{marra2005jgofs} and hence longer food chain \citep{banse1995zooplankton,barber2001qn}, explaining the carnivore dominated fisheries in NEAS \citep{shankar2019role} and planktivore dominated SEAS \citep{longhurst1990gd,shankar2019role}. 
	
	The southwest monsoon was found to be the most productive period \citep{Kumar.2000} however the observed primary productivity values were lower than predicted primary productivity owing to efficient grazing by mesozooplankton that kept diatom biomass in check instead of high levels of primary productivity as seen in coastal upwelling regions \citep{barber2001qn}. Similar to the zooplankton variability, the inter-annual variability of Chl-a is less in comparison to its seasonal variability \citep{shi2022phytoplankton} implying the inter-species relationship to be at play in shorter timescale with large and small phytoplankton dominating the SEAS \citep{shankar2019role}. It is inferred that along with the physico-chemical parameters, the biology of ocean determines the zooplankton-phytoplankton relationship and their biomass, respectively. This interdependency of planktons and the physico-chemical drivers shows up as strong intraseasonal and intra-annual variability in zooplankton biomass as demonstrated in \autoref{sec:seasonalcyclezss}. 

	\subsection{Current and biomass coherence}
	The annual variability shows that the contribution of this band to the time series of total biomass is very weak. It is low off Kanyakumari (standard deviation 3.64 mg~m$^{-3}$) and Okha (3.73 mg~m$^{-3}$), while it is stronger at rest of the basin, with strongest variability off Jaigarh (9.05 mg~m$^{-3}$). But the variation in chlorophyll biomass off Jaigarh is much lower than that of Kanyakumari. Owing to the above and coherence of current and ZSS, advection as a driver to upwelling and further as one of the cause for zooplankton growth is hypothesized and their relationship is explored. Wavelet coherence shows that the current and biomass have strong coherence off Kanyakumari (2019,2021), Kollam (2019, 2022) during May to late summer monsoon of 2019 with meridional current leading biomass, when the currents are reversing with monsoon (\cref{fig:biomasscurrentcoh}). As we go poleward along the slope, coherence exists but at different depth i.e, off Goa for 2019, the maximum coherence of meridional current with biomass is at 50 to 80 m and again below 110 m. The feature observed in annual filtered biomass off Goa is similar to the alongshore component \citep{nethery2007zm}, with the core of biomass and alongshore current lying at about 50 m. Further north, off Mumbai a shift in time of maximum coherence is observed occurring in winter monsoon at 80 m and below with zonal current leading biomass. During pre-summer monsoon upwelling sets as early as February in SEAS but only in May farther north along the coast \citep{banse1968hydrography} which results in shift in biomass coherence as we go poleward. Off Okha however,  present of coherence is seen throughout 2021 from 20 to 150 m with meridional current leading biomass which could be due to a deeper MLD in northern NEAS \citep{marra2005jgofs,shankar2016inhibition}. This has implications on the nature of zooplankton and fisheries found in regimes of EAS as we'll discuss in the following section.

    \subsection{Decoding the Arabian sea paradox: evolution of our understanding}
	
	While the zooplankton biomass was expected to have a seasonality, the transient nature of variability also explains why Arabian sea paradox was seen as such. In northern Arabian sea, the extended upwelling time leads to a longer and steady primary production, albeit weaker than the southern counterpart \citep{madhupratap1996lack, smith2005mesozooplankton}. Provided the zooplankton-phytoplankton interaction is based on primary production, this may lead to the zooplankton biomass to be consistent over season or longer i.e, weaker intraseasonal and intra-annual variability, for example off Goa, June 2019 to September 2020, (\cref{fig:intraseasonal,fig:intraannual}) the D215 remains at same depth, and this could be misinterpreted as constancy in zooplankton biomass leading to paradoxical conclusions. 	 
	
	Using the continuous data from ADCP backscatter, A22 showed not only there is a seasonality, but zooplankton-phytoplankton relationship can interact unexpectedly by negative zooplankton growth (dip in ZSS) when the phytoplankton bloom occurs during summer monsoon as was the case of Kollam. However, with the present study extending further boundary of EAS in south of Kollam, we come across Kanyakumari which matches with the paradox posited back in 1990s by \citet{madhupratap1992zooplankton, madhupratap1996lack, smith2005mesozooplankton} and studies from JGOFS cruise. Although the difference between the former and present paradox is fundamentally distinct, while former raised the question "how does zooplankton sustain it's population/growth in unviable conditions throughout year?", the present question is "why zooplankton doesn't seem to grow in tune with raising phytoplankton productivity during summer monsoon?". The answer to this question may lie in understanding the intricate interactions between phytoplankton, zooplankton, and fish, as well as the influence of monsoon currents on the availability of essential nutrients and rare elements that zooplankton need but cannot obtain from phytoplankton alone. \citep{shankar2019role} had found that if the upwelling is sufficiently strong, then phytoplankton tend to grow bigger in size and hence fish can compete directly with zooplankton on predation of phytoplankton, thereby limiting the zooplankton growth substantially which we see as a minuscule rise in ZSS (\cref{fig:zsschlclim} g2). For similar reasoning, a dip in ZSS is observed in peak summer monsoon off Kollam during August.

	\subsection{Conclusion}
	The results presented in this paper are based on the ADCP backscatter which is suitable for creating long-term time series of zooplankton biomass in open ocean. There are however, certain limitations to this approach. While the variation in depth is captured with in situ samples from MPN, the variation in season is not adequately addressed owing to the limitation of months when ADCP servicing cruises are undertaken. The west coast  cruises for ADCP servicing are planned for the monsoon transition months but may start as early as late September till December with few exceptions such as 2022 when it was carried out in March. Since the intraseasonal and intra-annual variability is almost double that of the annual one, the sampling done in particular season for biomass-backscatter comparison isn't sufficient. For a better approach to capture the  seasonal variation, more in situ samples are needed from the less explored seasons. 
	
	While we are able to infer the biomass information, any information regarding the size distribution of zooplankton and their contribution to ZSS is lost. In western Arabian sea, microzooplankton dominated the grazing processes by consuming approximately 71 \% of the primary production \citep{reckermann1997-kz,marra2005jgofs,landry2009-ti}. Mesozooplankton, in turn relied on microzooplankton for about 40 \% of their food \citep{landry2009-ti,hood2024nutrient}. However, the relative grazing importance of micro and mesozooplankton fluctuated seasonally and spatially, affecting the overall impact on phytoplankton biomass in a way that aligns with the theory of grazing control or trophic cascade \citep{ripple2016-nk} in the Arabian Sea \citep{marra2005jgofs,landry2009-ti}. To understand the intricate complexities of meso and microzooplankton interaction and their size distribution, multi-frequency, size-resolving backscatter data can be utilised.
	
	\section{Declaration of competing interest}
	The authors declare that they have no known competing financial interests or personal
	relationships that could have appeared to influence the work reported in this paper.
	
	\section{Acknowledgments} 
	The data were collected by xxx with fund provided under xxxx. The mooring programme is supported by INCOIS (Indian National Centre for Ocean Information Sevices, Hyderabad) and CSIR. We acknowledge the contribution of mooring division and ship cell of CSIR-NIO. 
	
	Ranjan Kumar Sahu expresses his acknowledgment to the Council of Scientific and Industrial Research (CSIR) for sponsoring his fellowship. Additionally, he extends his thanks to Ashok Kankonkar for providing the essential data, Rahul Khedekar for his data processing, Roshan D'Souza for his diligent work in biological data analysis. Their contributions were invaluable to the successful completion of this research.

\linespread{1.5}	
{\footnotesize 	\bibliographystyle{plainnat} % Choose a bibliography style
	\bibliography{bs_citations} % Specify your .bib file
}	
\newpage
\newgeometry{top=1in, bottom=1in} 

\linespread{1} 	
\begin{table}[htbp]

	{\footnotesize

		\captionsetup{justification=justified,font=footnotesize,skip=0.05\baselineskip} % Adjust the spacing above and below the caption
		\caption{ADCP deployment details at the locations. The temporal resolution is 1 hour, bin size(vertical resolution) 4 m. All ADCPs are operated at 153.3 kHz. The moorings are at a water column depth of ~950--1200 m on the continental slope and are serviced on yearly basis according to ship availability. The 6th column consists of Reference echo intensity (Er) for each beam, while the 7th column contains the corresponding RSSI conversion factor \citep{deines1999backscatter}.}
		\begin{adjustwidth}{0in}{0in} 
			\begin{tabular}{ccccccc}
				
				\toprule
				\multicolumn{1}{c}{}        & \multicolumn{2}{c}{Date}                                       & \multicolumn{2}{c}{Depth}                                                              & &          \\ 
				\midrule
				\multicolumn{1}{c}{\begin{tabular}[c]{@{}c@{}} Station \\ (Position; $^o$E,$^o$N) \end{tabular}} & \multicolumn{1}{c}{Deployment} & \multicolumn{1}{c}{Recovery} & \multicolumn{1}{c}{Ocean} & \multicolumn{1}{c}{ADCP} & \multicolumn{1}{c}{Er} & \multicolumn{1}{c}{Kc} \\
				\midrule
				\multirow{4}{*}{\begin{tabular}[c]{@{}c@{}} Okha\\ (67.47, 22.26)\end{tabular}}         & 01/10/2018                      & 01/12/2019                    & 996                        & 118                       & 37                          , 37                          , 37                          , 36 &                           0.42                        , 0.44                        , 0.42                        , 0.43                        \\
				& 01/12/2019                      & 04/12/2020                    & 1166                       & 312                       & 39                          , 36                          , 38                          , 36                          & 0.42                        , 0.44                        , 0.42                        , 0.43                        \\
				& 04/12/2020                      & 08/03/2022                    & 1021                       & 144                       & 41                          , 37                          , 38                          , 37                          & 0.42                        , 0.44                        , 0.42                        , 0.43                        \\
				& 08/03/2022                      & 01/01/2023                    & 1019                       & 142                       & 37                          , 38                          , 39                          , 36                          & 0.42                        , 0.44                        , 0.42                        , 0.43                        \\
				\midrule
				\multirow{5}{*}{\begin{tabular}[c]{@{}c@{}} Mumbai \\ (69.24, 20.01)\end{tabular}}        & 09/11/2017                      & 29/09/2018                    & 1025                       & 150                       & 36                          , 34                          , 39                          , 42                          & 0.40                        , 0.40                        , 0.40                        , 0.40                        \\
				& 29/09/2018                      & 29/11/2019                    & 1122                       & 125                       & 35                          , 36                          , 39                          , 42                          & 0.40                        , 0.40                        , 0.40                        , 0.40                        \\
				& 29/11/2019                      & 02/12/2020                    & 1143                       & 164                       & 37                          , 34                          , 39                          , 43                          & 0.40                        , 0.40                        , 0.40                        , 0.40                        \\
				& 02/12/2020                      & 06/03/2022                    & 1125                       & 142                       & 36                          , 34                          , 39                          , 42                          & 0.40                        , 0.40                        , 0.40                        , 0.40                        \\
				& 07/03/2022                      & 02/01/2023                    & 1103                       & 158                       & 37                          , 34                          , 40                          , 43                          & 0.40                        , 0.40                        , 0.40                        , 0.40                        \\
				\midrule
				\multirow{5}{*}{\begin{tabular}[c]{@{}c@{}} Jaigarh \\ (71.12, 17.53)\end{tabular}}       & 27/10/2017                      & 27/09/2018                    & 1039                       & 198                       & 32                          , 35                          , 33                          , 32                          & 0.45                        , 0.45                        , 0.45                        , 0.45                        \\
				& 27/09/2018                      & 30/10/2019                    & 1032                       & 164                       & 32                          , 35                          , 33                          , 31                          & 0.45                        , 0.45                        , 0.45                        , 0.45                        \\
				& 03/11/2019                      & 30/11/2020                    & 1142                       & 264                       & 32                          , 36                          , 33                          , 32                          & 0.45                        , 0.45                        , 0.45                        , 0.45                        \\
				& 30/11/2020                      & 05/03/2022                    & 1099                       & 119                       & 33                          , 36                          , 34                          , 32                          & 0.45                        , 0.45                        , 0.45                        , 0.45                        \\
				\midrule
				\multirow{5}{*}{\begin{tabular}[c]{@{}c@{}} Goa\\ (72.74, 15.17)\end{tabular}}          & 03/10/2017                      & 25/09/2018                    & 1000                       & 174                      & 35                          , 37                          , 34                          , 35                          & 0.44                        , 0.44                        , 0.40                        , 0.41                        \\
				& 25/09/2018                      & 16/10/2019                    & 969                        & 145                       & 38                          , 36                          , 36                          , 34                          & 0.44                        , 0.44                        , 0.40                        , 0.41                        \\
				& 16/10/2019                      & 29/11/2020                    & 966                        & 143                       & 44                          , 38                          , 36                          , 43                          & 0.44                        , 0.44                        , 0.40                        , 0.41                        \\
				& 29/11/2020                      & 03/03/2022                    & 985                        & 157                       & 35                          , 40                          , 35                          , 38                          & 0.44                        , 0.44                        , 0.40                        , 0.41                        \\
				& 03/03/2022                      & 05/01/2023                    & 984                        & 159                       & 35                          , 38                          , 35                          , 34                          & 0.44                        , 0.44                        , 0.40                        , 0.41                        \\
				\midrule
				\multirow{4}{*}{\begin{tabular}[c]{@{}c@{}} Udupi \\ (74.04, 12.5)\end{tabular}}         & 05/10/2017                      & 06/10/2018                    & 1028                       & 176                       & 44                          , 46                          , 29                          , 35                          & 0.45                        , 0.45                        , 0.45                        , 0.45                        \\
				& 06/10/2018                      & 18/10/2019                    & 1027                       & 179                       & 32                          , 38                          , 30                          , 36                          & 0.45                        , 0.45                        , 0.45                        , 0.45                        \\
				& 18/10/2019                      & 11/12/2020                    & 1018                       & 168                       & 33                          , 37                          , 31                          , 38                          & 0.45                        , 0.45                        , 0.45                        , 0.45                        \\
				& 11/03/2022                      & 06/01/2023                    & 1036                       & 155                       & 31                          , 32                          , 32                          , 33                          & 0.45                        , 0.45                        , 0.45                        , 0.45                        \\
				\midrule
				\multirow{5}{*}{\begin{tabular}[c]{@{}c@{}} Kollam \\ (75.44, 9.05)\end{tabular}}        & 07/10/2017                      & 08/10/2018                    & 1174                       & 200                       & 43                          , 55                          , 45                          , 43                          & 0.49                        , 0.50                        , 0.49                        , 0.50                        \\
				& 08/10/2018                      & 20/10/2019                    & 1160                       & 123                       & 49                          , 62                          , 46                          , 46                          & 0.49                        , 0.50                        , 0.49                        , 0.50                        \\
				& 20/10/2019                      & 13/12/2020                    & 1209                       & 176                       & 52                          , 61                          , 54                          , 55                          & 0.49                        , 0.50                        , 0.49                        , 0.50                        \\
				& 13/12/2020                      & 13/03/2022                    & 1129                       & 91                        & 49                          , 51                          , 46                          , 47                          & 0.49                        , 0.50                        , 0.49                        , 0.50                        \\
				& 13/03/2022                      & 08/01/2023                    & 1149                       & 164                       & 41                          , 48                          , 43                          , 41                          & 0.49                        , 0.50                        , 0.49                        , 0.50                        \\
				\midrule
				\multirow{6}{*}{\begin{tabular}[c]{@{}c@{}} Kanyakumari \\ (77.39,6.96)\end{tabular}}   & 16/11/2016                      & 08/10/2017                    & 1096                       & 252                       & 37                          , 36                          , 37                          , 37                          & 0.42                        , 0.44                        , 0.42                        , 0.43                        \\
				& 08/10/2017                      & 10/10/2018                    & 1055                       & 181                       & 32                          , 34                          , 38                          , 35                          & 0.45                        , 0.45                        , 0.45                        , 0.45                        \\
				& 10/10/2018                      & 22/10/2019                    & 1075                       & 180                       & 36                          , 34                          , 39                          , 36                          & 0.45                        , 0.45                        , 0.45                        , 0.45                        \\
				& 22/10/2019                      & 14/12/2020                    & 1060                       & 167                       & 33                          , 35                          , 36                          , 35                          & 0.45                        , 0.45                        , 0.45                        , 0.45                        \\
				& 14/12/2020                      & 14/03/2022                    & 1184                       & 287                       & 34                          , 36                          , 36                          , 35                          & 0.45                        , 0.45                        , 0.45                        , 0.45                        \\
				\bottomrule
			\end{tabular}
		\end{adjustwidth}
		\label{tab:table1}
	}	
\end{table}
\restoregeometry

\newpage

\begin{table}[htbp]
	
	{\footnotesize
		\captionsetup{justification=justified,font=footnotesize,skip=0.05\baselineskip,width=\textwidth} % Adjust the spacing above and below the caption
		\caption{\newline Volumetric samples of zooplankton of various stations. The sampling depth range is standardised for later years for bin range of 0--25m, 25--50m, 50--75m, 75--100m, 100--150m. The abbreviations are in the following manner: Okha (O), Mumbai (M), Jaigarh (J), Goa (G), Udupi (U), Kollam (K), Kanyakumari (KK); The number tags corresponds to particular cruise of a station.}
		\begin{adjustwidth}{0in}{0in} 
			\begin{tabular}{ccccccc}
				\toprule
				Sample number & Tag & Lat($^o$N)    & Lon($^o$E)   & Date & Time (IST) & Sampling depth range (m)      \\
				\midrule
				1-3         & G1  & 15.18      & 72.79      & 25 Sep 18                 & 452        & 50–25, 100–50, 150–100        \\
				4-6         & G2  & 15.16      & 72.71      & 25 Sep 18                 & 2108       & 50–25, 100–50, 150–100        \\
				7-10        & G2  & 15.16      & 72.71      & 25 Sep 18                 & 2137       & 40–20, 60–40, 80–60, 100–80   \\
				11-14       & J1  &            &            & 26 Sep 18                 & 2000       & 40–20, 60–40, 80–60, 100–80   \\
				15-17       & J2  &            &            & 27 Sep 18                 & 2000       & 50–25, 100–50, 150–100        \\
				18-21       & J2  &            &            & 27 Sep 18                 & 2100       & 40–20, 60–40, 80–60, 100–80   \\
				22-25       & M1  & 20         & 69.19      & 28 Sep 18                 & 2135       & 40–20, 60–40, 80–60, 100–80   \\
				26-27       & M1  & 20         & 69.19      & 28 Sep 18                 & 2205       & 50–25, 100–50                 \\
				28-29       & M2  & 20.01      & 69.2       & 29 Sep 18                 & 2035       & 50–25, 100–50                 \\
				30-33       & M2  & 20.01      & 69.2       & 29 Sep 18                 & 2057       & 40–20, 60–40, 80–60, 100–80   \\
				34-37       & U1  &            &            & 5 Oct 18                  & 2000       & 40–20, 60–40, 80–60, 100–80   \\
				38-40       & U1  &            &            & 5 Oct 18                  & 2100       & 50–25, 100–50, 150–100        \\
				41-43       & U2  &            &            & 6 Oct 18                  & 2000       & 50–25, 100–50, 150–100        \\
				44-47       & U2  &            &            & 6 Oct 18                  & 2100       & 40–20, 60–40, 80–60, 100–80   \\
				48-51       & K1  & 9.06       & 75.42      & 8 Oct 18                  & 421        & 40–20, 60–40, 80–60, 100–80   \\
				52-54       & K1  & 9.06       & 75.42      & 8 Oct 18                  & 449        & 50–25, 100–50, 150–100        \\
				55-56       & K2  & 9.04       & 75.4       & 8 Oct 18                  & 2027       & 50–25, 100–50                 \\
				57-60       & K2  & 9.04       & 75.4       & 8 Oct 18                  & 2045       & 40–20, 60–40, 80–60, 100–80   \\
				\midrule
				61-64       & G2  & 15.16      & 72.74      & 16 Oct 19                 & 829        & 50–25, 75–50, 100–75, 150–100 \\
				65-67       & G3  & 15.16      & 72.74      & 16 Oct 19                 & 1812       & 50–25, 75–50, 100–75          \\
				68-70       & K2  & 9.02       & 75.42      & 20 Oct 19                 & 840        & 50–25, 75–50, 100–75          \\
				71-74       & K3  & 9.04       & 75.43      & 20 Oct 19                 & 1934       & 50–25, 75–50, 100–75, 150–100 \\
				75-78       & KK1 &            &            & 22 Oct 19                 & 742        & 50–25, 75–50, 100–75, 150–100 \\
				79-82       & KK2 &            &            & 22 Oct 19                 & 1925       & 50–25, 75–50, 100–75, 150–100 \\
				83-86         & J1  &            &            & 30 Oct 19                 & 324        & 50–25, 75–50, 100–75, 150–100 \\
				87-89         & J2  &            &            & 4 Nov 19                  & 946        & 75–50, 100–75, 150–100        \\
				90-92         & M2  & 19.98      & 69.22      & 29 Nov 19                 & 1434       & 50–25, 75–50, 100–75          \\
				93-96         & M3  & 20.01      & 69.23      & 30 Nov 19                 & 958        & 50–25, 75–50, 100–75, 150–100 \\
				97-100        & O1  & 22.24      & 67.49      & 1 Dec 19                  & 937        & 50–25, 75–50, 100–75, 150–100 \\
				101        & O2  & 22.25      & 67.46      & 1 Dec 19                  & 1957       & 150-100                       \\
				\midrule
				102-105       & G3  & 15.68      & 73.22      & 28 Nov 20                 & 930        & 50–25, 75–50, 100–75, 150–100 \\
				105-108       & G4  & 15.32      & 73.22      & 29 Nov 20                 & 1558       & 50–25, 75–50, 100–75, 150–100 \\
				108-110       & J2  & 17.85      & 71.21      & 30 Nov 20                 & 1458       & 75–50, 100–75, 150–100        \\
				111-114       & J3  & 17.91      & 71.21      & 1 Dec 20                  & 1052       & 50–25, 75–50, 100–75, 150–100 \\
				115-118       & M4  & 20.03      & 69.38      & 2 Dec 20                  & 2016       & 50–25, 75–50, 100–75, 150–100 \\
				119.00        & O2  & 22.41      & 67.8       & 4 Dec 20                  & 953        & 150-100                       \\
				120-123       & O3  & 22.41      & 67.79      & 4 Dec 20                  & 2011       & 50–25, 75–50, 100–75, 150–100 \\
				124-127       & K3  & 9.11       & 75.72      & 12 Dec 20                 & 2335       & 50–25, 75–50, 100–75, 150–100 \\
				128-131       & K4  & 9.06       & 75.74      & 13 Dec 20                 & 1507       & 50–25, 75–50, 100–75, 150–100 \\
				132-134       & KK1 & 7.62       & 77.63      & 14 Dec 20                 & 1226       & 50–25, 75–50                  \\
				135-138       & KK2 & 7.62       & 77.63      & 14 Dec 20                 & 2047       & 50–25, 75–50, 100–75, 150–100 \\
				\midrule
				139-142       & G4  & 15.32      & 73.21      & 3 Mar 22                  & 823        & 50–25, 75–50, 100–75, 150–100 \\
				143-146       & G5  & 15.68      & 73.21      & 4 Mar 22                  & 1030       & 50–25, 75–50, 100–75, 150–100 \\
				147-150       & M5  & 19.99      & 69.23      & 7 Mar 22                  & 957        & 50–25, 75–50, 100–75, 150–100 \\
				151-154       & O3  & 22.24      & 67.5       & 8 Mar 22                  & 806        & 50–25, 75–50, 100–75, 150–100 \\
				155-158       & U3  & 12.5       & 74.04      & 12 Mar 22                 & 1156       & 50–25, 75–50, 100–75, 150–100 \\
				159-160       & K4  & 9.04       & 75.42      & 13 Mar 22                 & 1027       & 50–25, 75–50, 100–75          \\
				161-164       & KK3 & 6.97       & 77.4       & 15 Mar 22                 & 1220       & 50–25, 75–50, 100–75, 150–100
				\\ 
				\bottomrule
			\end{tabular}
		\end{adjustwidth}
		\label{tab:table2}
	}
\end{table}

\newpage
\begin{table}[t]
	
	{\footnotesize
		\captionsetup{justification=justified,font=footnotesize,skip=0.05\baselineskip,width*=\columnwidth} % Adjust the spacing above and below the caption
		\caption{\newline The mean, standard deviation at 40 and 104 m of zooplankton biomass (mg~m$^{-3}$), standard deviation of ZSS (gm~m$^{-2}$) and chlorophyll (mg~m$^{-3}$) at 7 mooring sites are tabulated along with the standard deviation of components of biomass variability, namely intraseasonal, intra-annual and annual.}
	\begin{adjustwidth}{0in}{0in} 
\begin{tabular}{ccccccccccc}
	& \multicolumn{2}{c|}{40 m biomass} & \multicolumn{2}{c|}{104 m biomass}        & \multicolumn{1}{c|}{}                                                                            & \multicolumn{5}{c}{standard deviation}              \\ \hline
	& Mean         & Std        & Mean   & \multicolumn{1}{c|}{Std} & \multicolumn{1}{c|}{\begin{tabular}[c]{@{}c@{}}decrease with\\ depth (40 m - 104m)\end{tabular}} & Chl  & ZSS  & Intraseasonal & Intra-annual & Annual \\ \hline
	Okha        & 230.42       & 22.84      & 151.68 & 25.58                    & 78.74                                                                                            & 0.25 & 1.93 & 64.26         & 63.78        & 22.38  \\
	Mumbai      & 272.86       & 34.95      & 182.24 & 30.34                    & 90.62                                                                                            & 0.13 & 2.9  & 70.74         & 83.52        & 41.58  \\
	Jaigarh     & 278.45       & 36.52      & 182.96 & 48.89                    & 95.49                                                                                            & 0.05 & 3.24 & 90.3          & 87.06        & 54.3   \\
	Goa         & 235.22       & 30.34      & 163.02 & 36.54                    & 72.2                                                                                             & 0.15 & 2.24 & 76.38         & 83.76        & 38.58  \\
	Udupi       & 247.81       & 34.37      & 169.37 & 38.8                     & 78.43                                                                                            & 0.55 & 2    & 77.22         & 100.86       & 41.64  \\
	Kollam      & 272.56       & 54.94      & 198.89 & 50.08                    & 73.67                                                                                            & 0.68 & 1.25 & 89.94         & 95.94        & 41.82  \\
	KanyaKumari & 207.07       & 30.42      & 167.63 & 20.89                    & 39.44                                                                                            & 0.51 & 0.67 & 71.88         & 52.62        & 21.84  \\ \hline
\end{tabular}
	\end{adjustwidth}
    \label{tab:table3}
    }
\end{table}

\newpage
\begin{figure}[htbp]
	\centering
	\includegraphics[width=0.6\textwidth]{/media/scilab/disk_ranjan/works/backscatter_wc/figures/adcp_moorings_new1.jpg} 
	\captionsetup{justification=justified,font=footnotesize,skip=0.05\baselineskip,width=0.8\textwidth}
	\caption{Map showing region of interest in eastern Arabian Sea. The slope moorings are
		deployed at $\sim$ 1000 m depth as shown in the bathymetry contour. Note the increase in shelf width as we go poleward along the coast. The mooring sites off Okha and Mumbai are in Northern EAS; Jaigarh and Goa in Central EAS while Kollam and Kanyakumari are at Southern EAS. Udupi is situated at the transition zone of Central and Southern EAS.}
	\label{fig:map}
\end{figure}

\newpage
\begin{figure}[htbp]
	\centering
	\includegraphics[width=0.5\textwidth]{/media/scilab/disk_ranjan/works/backscatter_wc/figures/backscatter_vs_biomass.png} 
	\captionsetup{justification=justified,font=footnotesize,skip=0.05\baselineskip,width=0.8\textwidth}
	\caption{The linear fit line of Biomass (log$_{10}$ scale) and backscattering strength (in dB). The linear fit line is within the error range of previous result of \citep{aparna2022seasonal} (contained 67 data points) onto which latest zooplankton volumetric sample data (159 data points) is appended. The regression equation is $y\ = (0.02 \pm\ 0.0025) \ x + (4.0144 \pm \ 0.2198) $ and correlation value of 0.54. The dashed green lines denote error range of plausible slope and intercept. From the linear equation, the upper and lower bound of error limit leads to an error bar of $\sim$  14 $ mg\ m^{-3}$. The first standard deviation of log$_{10}$(Biomass) is $\pm$ 0.49, which results in the backscatter range of 48.58 dB encompassing the entire backscatter range. It signifies the robustness of zooplankton biomass dependency on ADCP measured backscattering strength.}
	\label{fig:bstobm}
\end{figure}

\newpage

\begin{figure}[htbp]
	\centering
	\includegraphics[width=\textwidth]{/media/scilab/disk_ranjan/works/backscatter_wc/figures/biomass_daily_monthly.png} 
	\captionsetup{justification=justified,font=footnotesize,skip=0.05\baselineskip,width=\textwidth}
	\caption{The Daily and monthly averaged biomass for EAS moorings, north (top) to south (bottom). The black contours are marking of 175 mg~m$^{-3}$ biomass for Okha and Kanyakumari; 215 mg~m$^{-3}$  for Mumbai, Goa and Kollam. The biomass contours are distinct and different based on the physico-chemical parameters and the one that best explains seasonality at respective location.  The top 10 \% of data is discarded due to echo noise. The dashed line at 22 m marks the top-depth of first bin i.e, 24 m.}
	\label{fig:dailynmonthly}
\end{figure}

\begin{figure}[htbp]
	\centering
	\includegraphics[width=\textwidth]{/media/scilab/disk_ranjan/works/backscatter_wc/figures/climatology_biomass_ss_chl.png} 
	\captionsetup{justification=justified,font=footnotesize,skip=0.05\baselineskip,width=\textwidth}
	\caption{Monthly climatology of zooplankton biomass is shown in left panels for 7 locations, (top to bottom is southward). The D175 \& D215 are shown in solid lines; dashed line represents the depth of 23 C isotherm; oxygen contours are shown in dotted lines and labeled for each mooring. The right set of panel plots is showing ZSS (24--140 biomass integral) and chlorophyll climatology for corresponding locations.}
	\label{fig:zsschlclim}
\end{figure}

\begin{figure}[htbp]
	\centering
	\includegraphics[width=\textwidth]{/media/scilab/disk_ranjan/works/backscatter_wc/figures/aparna_ranjan_climatology_comparison.png} 
	\captionsetup{justification=justified,font=footnotesize,skip=0.05\baselineskip,width=\textwidth}
	\caption{Monthly climatology of zooplankton biomass is shown in left panels for 3 locations which were earlier used in \citep{aparna2022seasonal}; a1, b1 \& c1 is the biomass climatology for Mumbai, Goa and Kollam, d1 is for ZSS climatology (24--140 biomass integral), e1 is for chlorophyll biomass climatology; a2, b2, c2, d2 \& e2 is same but based on data from 2017 to 2023. The D215 is shown in solid line. The dashed line represents the depth of 23 $^o$C isotherm; oxygen contours are shown in dotted lines and labeled for each mooring.}
	\label{fig:zsschlclimcomp}
\end{figure}

\begin{figure}[htbp]
	\centering
	\includegraphics[width=\textwidth]{/media/scilab/disk_ranjan/works/backscatter_wc/figures/biomass_40m_104m.png} 
	\captionsetup{justification=justified,font=footnotesize,skip=0.05\baselineskip,width=\textwidth}
	\caption{The daily biomass at depth of 40 m and 104 m for all locations shown by grey and cyan curves. The black and blue lines shows the 30 day rolling averaged biomass at 40 and 104 m, respectively. Notice the bursts seen in the daily data ranging from few days to weeks.}
	\label{fig:compfourty}
\end{figure}

\begin{figure}[htbp]
	\centering
	\includegraphics[width=\textwidth]{/media/scilab/disk_ranjan/works/backscatter_wc/figures/west_coast_wavelet_40m_104m.png} 
	\captionsetup{justification=justified,font=footnotesize,skip=0.05\baselineskip,width=\textwidth}
	\caption{Wavelet power spectra (Morlet) of the 40 m (left panel) and 104 m (right panel) zooplankton biomass plotted against time as abscissa and period in days as ordinate. The wavelet power is in log$_2$ scale, the 95 \% significance is marked in black contours; the cross-shaded region falls under cone of influence. Vertical white lines separates years.}
	\label{fig:wave40104}
\end{figure}


\begin{figure}[htbp]
	\centering
	\includegraphics[width=\textwidth]{/media/scilab/disk_ranjan/works/backscatter_wc/figures/west_coast_wavelet_ss_scale.png} 
	\captionsetup{justification=justified,font=footnotesize,skip=0.05\baselineskip,width=\textwidth}
	\caption{Wavelet power spectra (Morlet) of zooplankton standing stock plotted against time as abscissa and period in days as ordinate. The wavelet power is in log$_2$ scale, the 95 \% significance is marked in black contours; The vertical white lines separates years. The right side panel shows the ZSS (24--120 m biomass integral) time series of 30 day rolling mean data (black) overlaid upon daily data (Grey). The 30 day rolling mean data of chlorophyll (solid blue) is plotted over its daily data (cyan).}
	\label{fig:wavess}
\end{figure}


\begin{figure}[htbp]
	\centering
	\includegraphics[width=\textwidth]{/media/scilab/disk_ranjan/works/backscatter_wc/figures/biomass_40m_2019_Kollam_l1.jpeg} 
	\captionsetup{justification=justified,font=footnotesize,skip=0.05\baselineskip,width=\textwidth}
	\caption{Comparison between mean-removed daily biomass time series at 40 m and the distinct components of variability off Kollam for 2019. The biomass units are mg~m$^{-3}$. An increase in biomass is noticed from May onward and lasting till late monsoon with weeks of low biomass during August due to contribution from each component of variability. The cyan curve is sum of all low frequency components above 30 days, i.e, annual, intra-annual and 30 to 90 days intraseasonal variability.}
	\label{fig:variability}
\end{figure}

\begin{figure}[htbp]
	\centering
	\includegraphics[width=1.05\textwidth]{/media/scilab/disk_ranjan/works/backscatter_wc/figures/annual_300_400_451_1.jpeg} 
	\captionsetup{justification=justified,font=footnotesize,skip=0.05\baselineskip,width=\textwidth}
	\caption{The biomass variation occurring in annual band (300 to 400 days). Owing to the presence of monsoon, there is a variation driven by associated upwelling (downwelling) processes in summer (winter) monsoon. and The horizontal black and blue lines is for 40 and 104 m respectively; vertical black lines separate the years. The dashed line at 22 m marks the top-depth of first bin i.e, 24 m and solid orange curves denotes D215 (D175 off Okha and Kanyakumari)}
	\label{fig:annual}
\end{figure}



\begin{figure}[htbp]
	\centering
	\includegraphics[width=\textwidth]{/media/scilab/disk_ranjan/works/backscatter_wc/figures/intraannual_100_250_351.jpeg} 
	\captionsetup{justification=justified,font=footnotesize,skip=0.05\baselineskip,width=\textwidth}
	\caption{The biomass variation occurring in 100 to 250 days period (between the seasons and within a year record or intra-annual band) is obtained using a band pass filter. The horizontal black and blue lines is for 40 and 104 m respectively; vertical black lines separate the years. The dashed line at 22 m marks the top-depth of first bin i.e, 24 m and solid orange curves denotes D215 (D175 off Okha and Kanyakumari). }
	\label{fig:intraannual}
\end{figure}

\begin{figure}[htbp]
	\centering
	\includegraphics[width=\textwidth]{/media/scilab/disk_ranjan/works/backscatter_wc/figures/intraseasonal_30_90_181.jpeg} 
	\captionsetup{justification=justified,font=footnotesize,skip=0.05\baselineskip,width=\textwidth}
	\caption{Biomass variation found in the scale of 30 to 90 days  period (Intraseasonal band as it is within a season) is obtained using a lanczos band pass filter. The horizontal black and blue lines is for 40 and 104 m respectively; vertical black lines separate the years. The dashed line at 22 m marks the top-depth of first bin i.e, 24 m.  Intraseasonal variability is seen throughout the record and the variability is stronger during August to November.}
	\label{fig:intraseasonal}
\end{figure}


\begin{figure}[htbp]
	\centering
	\includegraphics[width=0.9\textwidth]{/media/scilab/disk_ranjan/works/backscatter_wc/figures/Current_biomass_and_coherence_2019.png} 
	\captionsetup{justification=justified,font=footnotesize,skip=0.05\baselineskip,width=\textwidth}
	\caption{Current (zonal/meridional) and biomass wavelet coherence plotted alongside of annual filtered biomass and current. Either zonal or meridional current is considered depending on coherence and if current leads biomass. The solid contour encompasses greater than 0.5 coherence and phase is plotted with north as reference, +ve (-ve) x-axis is current leading (lagging) biomass. Left (Right) panel is for biomass (current). Although the annual biomass variability is weak and contributes less to the total biomass time series, upward phase propagation is seen implying upwelling favorable conditions leading to biomass growth. The solid orange curves denotes D215 (D175 off Okha and Kanyakumari)}
	\label{fig:biomasscurrentcoh}
\end{figure}



\end{document}
